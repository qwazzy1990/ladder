\section{Introduction to the Problem}
Listing problems are common problems in cambinatorics. In general, listing problems 
focus on enumerating the objects of a given finite set in some specific order. The listing problem in this thesis 
will be termed \emph{The Canonical Ladder Listing Problem}. The problem is stated as follows: Let 
$S_{N}$ be the set of all $N!$ permutations of order $N$. To see $S_{4}$ please refer to table \ref{table:aa}. 
Let $\pi$ be a permutation from $S_{N}$. 
Let \emph{the canonical ladder} be a unique ladder from each permutation's $OptL\{\pi\}$. Let $CanL\{\pi_{N}\}$ be the set of all 
canonical ladders for each $\pi$ in $S_{N}$. Let $L_{i}$ be some arbitrary canonical ladder from $CanL\{\pi_{N}\}$. A \emph{change} 
is defined as the insertion or deletion of one or more bar(s) to get from $L_{i}$ to $L{j}$ or the relocation of one or more bars 
in $L_{i}$ to get to $L_{j}$. The \emph{relocation} of a bar is defined as moving a bar from a given row and column 
to a new row and/or column in the ladder under the following conditions.
The relocation cannot be a right/left swap operation. If the relocation of the bar moves the bar to a new row, but not a new column, then the endpoint 
of the bar being moved must cross the endpoint of a bar not being moved. To see examples of the relocation of a bar please refer to Fig.\ref{fig:bar-relocation}   
The canonical listing problem asks, given all $S_{N}$, is there an efficient way to generate $CanL\{\pi_{N}\}$? Efficiency is defined as 
using minimal change to transition from $L_{i}$ to $L_{j}$. For example, let $N=4$, then $|S_{N}|=24$ 
Since each permutation has at least one ladder in its respective $OptL\{\pi\}$, then $|CanL\{\pi_{4}\}=24|$. $CanL\{\pi\}$ consists 
of a canonical representative from each $OptL\{\pi\}$. 

\begin{center}
\begin{table}[ht]
    \caption{Table for all 4!, 24, permutations of order 4} 

    \centering
       \begin{tabular}{|c |c |c |c |}
        \hline
        1234 & 1243 & 1324 & 1342 \\ \hline
        1423 & 1432 & 2143 & 2134 \\ \hline 
        2314 & 2341 & 2413 & 2431 \\ \hline 
        3124 & 3142 & 3214 & 3241 \\ \hline 
        3412 & 3421 & 4123 & 4132 \\ \hline 
        4213 & 4231 & 4312 & 4321 \\ \hline 
        
    \end{tabular}
    \label{table:aa}
 \end{table}
\end{center}


\begin{theorem}
    In order to transition from canonical ladder $L_{i}$ to canonical ladder $L_{j}$, at least one bar has to be added or 
    removed from $L_{i}$ or at least one bar has to be relocated in $L_{i}$.
\end{theorem}
\begin{proof}
    We begin this proof by contradiction. Suppose $L_{i}$ is some arbitrary canonical ladder for some arbitrary permutation, $\pi$, of 
    order $N$. Suppose that $L_{j}$ is the next canonical ladder in the set of canonical ladders for some other arbitrary permutation $\pi+1$ 
    of order $N$. Suppose a bar does not need to be added or removed from $L_{i}$ to get to $L_{j}$ nor does a bar need to be relocated to get to $L_{i}$ to 
    $L_{j}$. We know that each bar in $L_{i}$ uninverts a single inversion in $\pi_{i}$. We know that each bar in $L_{j}$ 
    uninverts a single inversion in $\pi_{j}$. We know that $\pi_{i} \neq \pi_{j}$. Therefore we know that $L_{i} \neq L_{j}$. Two ladders corresponding to 
    two different permutations differ from each other in three ways. The first way is by the number of lines, the second way is by the number of 
    bars, and the third way is the location of bars. Note that two ladders can differ in more than one of the three ways. In the case 
    of $L_{i}$ and $L_{j}$, they have the same number of lines seeing as they are ladders of order $N$. Therefore they cannot differ in terms 
    of the number of lines. We also assumed that $L_{i}$ and $L_{j}$ have the same number of bars and the same location of bars. Which means 
    that the ladders are the same. But we already stated that $L_{i} \neq L_{j}$. Therefore we have a contradiction. Which means that 
    either a bar needs to be added/removed from $L_{i}$ to get to $L_{j}$ or a bar needs to be relocated in $L_{i}$ to get to $L_{j}$. End of proof.
\end{proof}


\begin{figure}[!htp]
    \centering
    \begin{tikzpicture}
        \node at(.5, 2.8){\small{$x$}};
        \draw(0, 0) to (0, 3);
            \draw(0, 2.5) to (1, 2.5);
            \draw(0, 1.5) to (1, 1.5);
        \draw(1, 0) to (1, 3);
            \draw(1, 2) to (2, 2);
            \draw(1, 1) to (2, 1);
        \draw(2, 0) to (2, 3);
            \draw(2, .5) to (3, .5);
        \draw(3, 0) to (3,3);

        \draw[->, line width = .3mm](-.5, -.5) to (-1.5, -1.5);
        \draw[->, line width =.3mm](3.5, -.5) to (4.5, -1.5);

        \draw(-2.5, -1.8) to (-2.5, -4.8);
            \draw(-2.5, -3) to (-1.5, -3);
                \node at(-2, -3.8){\small{$x$}};
            \draw(-2.5, -4) to (-1.5, -4);
        \draw(-1.5, -1.8) to (-1.5, -4.8);
            \draw(-1.5, -2.5) to (-.5, -2.5);
            \draw(-1.5, -3.5) to (-.5, -3.5);
        \draw(-.5, -1.8) to (-.5, -4.8);
            \draw(-.5, -4) to (.5, -4);
        \draw(.5, -1.8) to (.5,-4.8);

        
        \draw(2.5, -1.8) to (2.5, -4.8);
            \draw(2.5, -3.3) to (3.5, -3.3);
                \node at(5, -2.1){\small{$x$}};
            \draw(4.5, -2.3) to (5.5, -2.3);
        \draw(3.5, -1.8) to (3.5, -4.8);
            \draw(3.5, -2.8) to (4.5, -2.8);
            \draw(3.5, -3.8) to (4.5, -3.8);
        \draw(4.5, -1.8) to (4.5, -4.8);
            \draw(4.5, -4.3) to (5.5, -4.3);
        \draw(5.5, -1.8) to (5.5,-4.8);


    \end{tikzpicture} 
    \caption{Example of relocating bar $x$}
    \label{fig:bar-relocation}   
\end{figure}

In this thesis, two listing algorithms are used to list the canonical 
ladders for each $CanL\{\pi_{N}\}$. The first of these listing algorithms 
is a modification of the Steinhaus-Johnson-Trotter permutation listing algorithm. 
The second listing algorithm is influenced by Effler and Ruskey's algorithm in their paper 
A CAT Algorithm for Generating Permutations with a Fixed number of Inversions. 
It is termed the cyclic-bar algorithm. Both of these algorithms will 
be described, explained and analyzed throughout the remainder of the chapter.\par

Before proceeding, the selection process for choosing the canonical ladder will be explained.
In general, the canonical represeantive from $OptL{\pi_{N}}$ is chosen based on a tree structure. 
The root of the tree is the only ladder from $OptL\{\pi=(1, 2, \dots  ,N)\}$.
Proceeding from the root ladder, for every canonical ladder $L_{j}$, it 
is chosen based on the minimal amount of change required to get from $L_{i}$ where $j=i+1$. 
Since the minimal amount of change is defined as the insertion or deletion of one bar 
or the relocation of a bar, then the canonical representative, $L_{j}$, is equal 
to $L_{i}$ plus the minimal amount of change. The exception to this rule is 
in the cyclic-bar algorithm in which transitioning between trees in the forest 
structure generated by the algorithm.

