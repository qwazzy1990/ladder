
\subsection{Analysis}
    \subsubsection{Introduction}
    From looking at the table in the results section, it is cear that the modified SJT algorithm performs 
    better than the Cyclic Inversion algorithm. The reson(s) for this disparity in performance 
    will be analyzed. Following this analysis, areas of application and practical relavence for the Listing Problem 
    will be discussed along with concluding remarks.

\subsubsection{Performane Analysis}
    As $N \geq 9$ there is a noticeable difference between the runtimes of the two algorithms by a sizable order 
    of magnitude. Cleary the modified SJT algorithm performs better than the Cyclic Inversion algorithm. The reason(s) 
    for this improved performance are the following. Firstly, the time complexity of the two algorithms are different. 
    The time complexity for the modified SJT algorithm is $(N!)N$. The time will be proven in the following lemma.
    \begin{lemma}
        The time complexity for the modified SJT algorithm is $O((N!)N)$
    \end{lemma}
    \begin{proof}
        The $N!$ factor is fairly straightforward, the algorithm creates all $N!$ ladders in $CanL\{\pi_{N}\}$ which 
        accounts for the $N!$ factor. The $N$ factor is a result of the second for loop found in the algorithm. The first for loop 
        fount in the modified sjt function runs $(N-1)$ times each time the modified SJT function is called, however on each 
        iteration of this for loop a ladder is listed, therefore the runtime of this for loop is accounted for by the $N!$ factor. However, 
        the second for loop in the helper SJT function runs at worst, $N-1$ times before listing a ladder. This worst case 
        is when the $K=2$ route needs to have a bar inserted or removed. Therefore, this second for-loop accounts for the $N$ factor 
        in the time complexity. Thus, the time complexity of the modified SJT algorithm is $O((N!)N)$.
    \end{proof}

    On the other hand, the time complexity of the Cyclic Inversion algoirthm is $O((N!)N^{2})$. The time complexity for the Cyclic 
    Inversion Algorithm will be proven in the following lemma.

    \begin{lemma}
        The time complexity for the Cyclic Inversion algorithm is $O((N!)N^{2})$
    \end{lemma}
    \begin{proof}
        The $N!$ factor is fairly straightforward, the algorithm creates all $N!$ ladders in $CanL\{\pi_{N}\}$ which accounts 
        for the $N!$ factor. The $N^{2}$ factor is a result of the for loop that is executed when $2 \leq K < N$. This for loop runs 
        from $1$ to $K$ for each value of $K$. Thus, the for loop is executed $1 + 2 + 3 + 4, ... + N-1$ times. This summation 
        is equal to $((N-1)N-2)/2$ which is reduced to $N^{2}$. Therefore the for-loop when $2 \leq K < N$ accounts for the $N^{2}$
        factor.
    \end{proof}

    \subsubsection{Application(s)}
    The applications for generating $CanL\{\pi_{N}\}$ are currently unknown insofar as this problem has yet to be solved to my knowledge. 
    However, if I am to be granted some speculation, I could provide some hypothetical scenarios in which listing $CanL\{\pi_{N}\}$ could 
    be of interest. The first hypothetical application could be to model an \emph{oblivious sorting system} for $N!$ permutations. An oblivious 
    sorting system is a system such that the sorting operations are done irrespective of the data being passed to the system. Recall 
    that a bar in a ladder simply swaps two adjacent elements in a permutation. Due to the static nature of each ladder, the swap operation 
    resulting from two elements in a permutation crossing a bar is unchanging. 
    Seeing as each ladder in $CanL\{\pi_{N}\}$ sorts the corresponding permutation of order $N$, one can implement all of $CanL\{\pi_{N}\}$ 
    for some arbitrary $N$ value and then pass each permutation of order $N$ through its respective ladder from $CanL\{pi_{N}\}$ thus 
    resulting in each permutation being ordered. The ladders from $CanL\{\pi_{N}\}$ only need to be generated once and saved. Once this is done 
    a permutation can be passed to the correct ladder and it can be sorted by having each of its elements pass through the ladder. 