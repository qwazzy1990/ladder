\subsection{Procedure}

Thus far, the problem has been introduced and the required terminologt has been defined. Recall that there are two 
changes; the insertion/deletion of bars or repositioning bars.
However, there has yet to be discussion regarding the two 
listing algorithms. In the procedure section 
we look at the pseudo code for each of the algorithms and explain what 
each of the algorithms are doing. The goal is to transition from 
$L_{i}$ to $L_{i+1}$ with minimal change, which means adding or removing 
the least number of bars to get from $L_{i}$ to $L_{i+1}$ or relocating 
the least number of bars to get from $L_{i}$ to $L_{i+1}$.\par 

The reason that the modified SJT and CI algorithms were chosen is because they allow 
for minimal change from $L_{i}$ to $L_{i+1}$. While conducting this research, modifications 
to the permutation listing algorithms mentioned in chapter one were applied. Recall that 
these listing algorithms were Zaks, Heaps, and Lexicographic. These listing algorithms 
did not allow for minimal change when transitioning from $L_{i}$ to $L_{i+1}$.\pagebreak
%%section SJT
\subsubsection{Steinhaus-Johnson-Trotter}


\begin{algorithm}
  \begin{algorithmic}[1]
    \Function{modifiedSjt}{$N=[1, 3, 4, \dots]$, $Ladder[2N-3][N-1]$, $Arr[N-1]$, $Direction[N]$}


      \State $print(Ladder)$

      %%base case
      \If{$globalCount = n!$}
        \State return
      \EndIf

     
      \State $dir \gets direction[n-1]$

      %%swap the nth element n-1 times
      \For{$i \gets 0$,$i < n-1$, $i \gets i+1$}
        
        \State $arr[n-1] \gets arr[n-1]+1$
        \If{$dir = left$}
            \State $row \gets (n-1) + (i+1)$
            \State $col \gets row$
            \State $ladder[row][col] \gets 1$
        \Else
            \State $row \gets i$
            \State $col \gets row$
            \State $ladder[row][col] \gets 0$
        \EndIf
        \State $globalCount \gets globalCount+1$
        \State $print(Ladder)$

      \EndFor
      \State $direction[n-1] \gets !direction[n-1]$
      \State $HELPERSJT(n-1, ladder, arr, direction)$
      \State $MODIFIEDSJT(n,  ladder, arr, direction)$

    \EndFunction
  \end{algorithmic}
\end{algorithm}

%%helper algorithm
\begin{algorithm}
  \begin{algorithmic}[1]
    \Function{helpersjt}{$N-1=[1, 3, 4, \dots]$, $Ladder[2N-3][N-1]$, $Arr[N-1]$, $Direction[N]$}

      \State $c1 \gets 0$
      \State $c2 \gets 1$
      \For{$i \gets n-1$, $i \geq 0$, $i \gets i-1$}
        \If{$arr[i] < i$}
        \State $globalCount \gets globalCount + 1$

          \If{$dir[i] = LEFT$}
            \State $row \gets (i + c2) - c1 - arr[i]$
            \State $col \gets (i - 1) - arr[i]$
            \State $ladder[row][col] \gets 1$

          \Else
            \State $row \gets  arr[i] + (2(countTwo))$
            \State $col \gets arr[i]$
            \State $ladder[row][col] \gets 0$
          \EndIf
          \State $arr[i]\gets arr[i+1]$
          \State return
        \Else 
          \State $arr[i] \gets 0$
          \State $direction[i] \gets !direction[i]$
        \EndIf
        $c1 \gets c1 - 1$
        $c2 \gets c2 + 1$
      \EndFor
      \EndFunction
  \end{algorithmic}
\end{algorithm}

Let the \emph{identity ladder} be the ladder for the sorted permutation from $[1 \dots N]$.
Let the initial conditions of the algorithm be the fallowing. The $Ladder= 2D array$,  
let $n \geq 1$, let $arr$ be set to zero for all indexes. Let $dir$ be set to false 
for all indexes. The principles of the algorithm are the following, if the direction for a 
given route is false, then bars will be added for that given route, from right to left, bottom to top, until no more bars can be added.
If the direction for a given route is true, then bars will be removed for that given route, left to right, top to bottom, until 
no more bars can be removed. Let $K$ be the value of some given route where $1 < K <= N$. 
The number of bars for a given route is $0 < K < N$. This is because the maximum number of inversions 
the $kth$ element can make is $K-1$, therefore the $kth$ route can have at most $N-1$, if $K=N$, and 
at least $1$ bar since element $1$ has no route. Once all the bars for the $Kth$ route have been added 
or removed, the direction for the $Kth$ route is switched, indicating that its bars will be removed if they 
were added, or added if they were removed.\par 
\begin{theorem}
  The number of rows required for the ladder data-structure is $2N - 3$ and the number of columns required for 
  the ladder is $N-1$.
\end{theorem}
\begin{proof}
  The number of columns is fairly straighforward. Seeing as there are always $N$ elements in $\pi_{N}$, 
  a column represents a gap between lines in the corresponding ladder-lottery. Let $Line_{i}$ be a vertical line in a ladder-lottery 
  with some element in $\pi_{N}$ at the top of the line and the $ith$ element in $\pi_{N}$ be 
  at the bottom of $Line_{i}$. There are $N$ lines in the ladder-lottery, a column in the ladder data-structure
  simply represents a gap between two adjacent lines in the ladder lottery.\par 
  The number of rows for the ladder data-structure is calculated a follows, given $\pi_{N}$, the minimal 
  number of rows required is when $\pi_{N}$ is sorted. In this case there are zero rows because there are 
  zero bars added to the ladder. This ladder is $L_{N_{ID}}$ and is 
  the first ladder in $CanL{\pi_{N}}$. When a bar is added to the ladder it can be added to an already existing row 
  or to a new row. Since $L_$
\end{proof}
The calculations for the row and column for the bar 
depend on several factors. The first factor is whether the row and column is being calculated for $K=N$ or 
if $K < N$. If $K=N$, then the row and column are calculated using the main function, modifiedSJT.
If the row and column are being calculated for $K < N$, then the  row and column are calculated in the 
function helpersjt. The row and column are calculated differently depending on which function they are 
being calculated in. In the case of sjtmodified, the row is calculated as follows. Assuming a zero indexed array,  when adding bars 
for the route $K=N$, Let $I$ be the current 
value of the counter in the for-loop. The value of $I$ is one less than the $Mth$ bar to 
be added to the route. For example, if $i=0$, then the first bar of the $K=Nth$ route is
to be added. The row is equal to $N-i$ - 2. For example, if $K=N=5$ and the first bar of the 
fifth route was going to be added to the ladder, then $I=0$ and the row is the third row, seeing 
as $N-0-2=5-0-2=3$. The column is equal to the row because the bars are added right to left. 
For an example of adding bars of the $K=N'th$ route to the ladder see figure -- 
when removing a bar from the $K=Nth$ route, the row and column are simply equal to $I$. 
For example, if the first bar of the $K=Nth$ route is to be removed, said bar exists at 
row zero, column zero which is the value of $I$ on the first iteration of the for loop.\par 
Helpersjt is used when $K<N$. The array in helpersjt is used to keep track of the number of bars
that have been added or removed for a given $1< K <N$. Recall that the number of bars for any given route 
$K$ at any given time is $[0 \dots k-1]$. Therefore, when the number of bars for route $K=K-1$, 
the $arr[K-1]$  gets set to zero, and the $direction[K]$ gets switched indicating that all the bars of the 
$Kth$ route have to be removed if they were added, or added again if they were removed. Assuming a zero indexed 
array, when adding bars for route $K$ and the number of bars for route $K=[0 \dots K-2]$, then the row is calculated 
as follows. 

\begin{theorem}
  Let $arr$ be a zero indexed array. Let $arr[K-1]$ represent the number of bars for route $K$ that are currently in the 
  ladder. Let $L_{i}$ be a two dimensional, zero indexed array representing the current ladder.
  Let $1 < K < N$. The the row for the current bar to be added for routr $K$ is $Row=(N-2) + (N-K) - arr[K-1]$.
\end{theorem}
\begin{proof}
  It must be noted that we are listing only root ladders. So when transitioning from 
$L_{i}$ to $L_{i+1}$ in $CanL{\pi_{N}}$ both are root ladders. Recall that the root ladder is the ladder such that no  
route of any lesser value in $\pi$ has crossed the route of a greater value. With this in mind, one can say that the 
number of rows required for the $Nth$ value is $N-1$ seeing as the $Nth$ value can have at most $N-1$ bars in its route. 
Since bars are added right to left, bottom, up, then the first bar of route $K$ will be added to the row 
just  below the last bar of the previous route. The reason $N-2$ is added is because the route of the last row of the route of the $Nth$ element is 
at row $N-2$ in $L$. When the first bar is to be added, the $arr[k-1]=0$. If $K$ is one less than $N$ then 
the first bar of $K$ will be added one row below the last bar of $N$. If $K$ is two less than $N$ then the first bar 
of $K$ will be added two rows below the last bar of $N$, etc. The first bar of the $Kth$ route 
will be added to the 
\end{proof}
