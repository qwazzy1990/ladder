\subsection{Data Analysis}
The data will be analyzed in two parts. The first operation will be analyzed independent 
of the second operation, the second operation will be analyzed independent of the first operation 
and lastly, the two operations will be analyzed together.\par 
\subsubsection{Operation One}
For the first operation, which is the addition/removal of a bar, it is clear that the SJT Gray Code 
is the most efficient. In fact, there is no more efficient Gray code when it comes to the first operation.
This is true because every ladder corresponds to some permutation, $\pi_{N}$, which entails 
that there is at least one inversion added or removed when comparing two permutations of size $N$. 
Let these permutations be denoted as $\pi_{x}$ and $\pi_{y}$.
Thus, for each corresponding ladder from each permutation's $OptL\{\pi\}$, there must be at least 
one bar added or removed from $L_{x}$ to $L_{y}$. Otherwise they would be the same ladder corresponding 
to the same permutation. Notice how the average number of additions/removals of bars for the SJT algorithm 
stays 1 regardless of whether the root ladder is chosen or the optimal ladder is chosen. With the other three 
Gray Codes, the average number of addition/removal of bars increases as $N$ increases, yet there is no 
difference between choosing the root ladder or the optimal ladder as the canonical representative.
The rate of growth will be analyzed for these other three Gray Codes. The second best Gray Code for operation 
one is the lexicographic Gray Code. The values are
1.400000, 1.739130, 1.915966, 1.979138. These values are the same whether choosing the root ladder or 
the optimal ladder as the representative. As $N$ increased from three to six, 
the average number of insertions or deletions of bars was 1.758558 and the variance was 0.067458.
This suggests that the rate of growth for the number of insertions/deletions of bars moderately increases when $N$ increases whilst 
using Heaps algorithm. The third best Gray Code for operation one is Heaps Gray Code.
The values are 1.800000, 1.956522, 2.126050, 2.146036.
These values are the same regardless of the canonical representative. As $N$ increased from three to six, 
the average number of insertions or deletions of bars was 2.007152 and the variance was 0.026300.
This suggests that the rate of growth for the number of insertions/deletions of bars increases at a slower rate than the lexicographic Gray Code 
when $N$ increases whilst using Heaps algorithm. The worst Gray Code for operation one 
is Zaks Gray Code. The values are 1.8000000, 2.347826, 2.605042, 2.691238. These values 
are the same regardless of the canonical representative chosen. As $N$ increased from three to six, 
the average number of inserstions or deletions of bars was 2.361026 and the variance was 0.161169.
This suggests that the rate of growth for the number of insertions/deletions of bars increases at the fastest rate 
when $N$ increases whilst using Zaks algorithm.\par 
When it comes to operation one, it is clear that SJT is the best Gray Code. In terms of the second best Gray Code, 
it is debatable whether lexicographic or Heaps is better. Although lexicographic produced better numbers in the data set, the variance and mean is greater than Heaps as $N$ increased. 
On the other hand, it appears that as $N$ increases, the difference between the average number of insertions/deletions decreases with lexicographic Gray Code. The difference 
between $N=5$ and $N=6$ for lexicographic is 0.063172 whereas with Heaps it is 0.019986. This implies that as $N$ gets larger 
Heaps will outperform lexicographic. Zaks algorithm performed the worst in the data set and had the highest 
degree of variance as $N$ increased from three to six. Also, the difference between $N=5$ to $N=6$ is the largest 
when using Zaks algorithm, with a difference of 0.086196 providing further evidence for Zaks being the worst of the four Gray Codes when it comes to 
operation one.  
\subsubsection{Operation Two}
For the second operation, which is the swapping of two bars, there is a difference between choosing the root ladder or the optimal ladder as 
the canonical representative. However, choosing the optimal ladder as the representative did not make a difference for the SJT Gray Code.
This means that it is better to choose the optimal ladder when selecting the canonical representative for Heaps, Zaks and lexicographic, but 
not for SJT. If the root ladder is selected as the representative, then the order of best to worst Gray Code in terms of the data set is 
Heaps, SJT, Zaks, Lex. However, since three of the four perform better when the optimal ladder is selected, the remainder of the analysis 
for operation two will be done using the optimal ladder as the representative. The best algorithm, when it comes to operation two, is 
Zaks algorithm with the values for $N=[3 \dots 6]$ being 0.000000, 0.000000, 0.000000, 0.000000. The reason Zaks does not have any swap operations 
is becuase of the suffix reversal. When a suffix is reversed, every permutation in said suffix becomes uninverted and every non-inversion 
becomes inverted. Since the swap operation on $L_{i}$ to get to $L_{i+1}$ swaps bars that exist in $L_{i}$ and $L_{i+1}$, then removing 
said bars by reversing a suffix, implies the bars cannot be swapped. For example, given a suffix $"\dots2431"$, the inversion set is $\{(2,1),
(4,1), (3,1), (4,3)\}$ and the non-inversion set is $\{(4,2), (3,2)\}$. When the suffix is reversed, the resulting suffix is $\dots1342$ with the 
inversion set $\{(4,2), (3,2)\}$ and the non-inversion set $\{(2,1), (4,1), (3,1), (4,3)\}$. This implies that the resulting ladders would have no 
bars in common, thus resulting in no swap operations. The second best Gray Code was Heaps Gray Code with the values of 0.000000, 0.043478, 
0.058824, 0.054242 for $N=[3 \dots 6]$. The average number of swaps was 0.039136 when $N$ increased from three to six and the variance was 
0.000722. Also, as $N$ increased, the difference between the number of swaps decreased, implying that as $N$ gets large, Heaps algorithm performs 
better in terms of the average number of swaps. In fact, when $N=5$ the average number of swaps is greater 
than when $N=6$. It is possible that the average number of swaps will decrease as $N$ gets larger when 
using Heaps algorithm. Although this can only be true to a certain point, seeing as there cannot be a negative number of swaps.
The third best Gray Code was lexicographic Gray Code with the values 0.000000, 0.043478, 0.058824, 0.063978. 
The average number of swaps was 0.041570 when $N$ increased from three to six, and the variance was 
0.000844. The first three values are the same as with Heaps Gray Code, but notice that when $N=6$, the average 
number of swaps increased from $N=5$ for the lexicographic Gray Code, whereas it decreased for Heaps Gray Code.
Therefore lexicographic Gray Code performed worse than Heaps algorithm, but only marginally. The worst Gray Code for 
the second operation is SJT with the values of 0.200000, 0.173913, 0.134454, 0.109875. The average number of 
swaps when $N$ increased from three to six was 0.15460 and the variance was 0.001613 when using SJT Gray Code.
Notice how that when $N=6$, the average number of swaps is roughly twice as large as for lexicographic or Heaps.
However, also notice that when $N=6$, the average number of swaps is less than when $N=5$ 
for the SJT algorithm. Like with Heaps algorithm, this means it is possible that the average number 
of swaps will decrease for SJT as $N$ gets large. Perhaps at large values of $N$, SJT will outperform 
lexicographic and even Heaps when it comes to operation two. Yet when looking strictly at the data set, 
SJT performed the worst when it came to operation two.

\subsubsection{Operations One and Two}
In order to analyze operations one and two together, the optimal ladder will be used as the canonical
representative. This is because the performance of operation one did not change based on the 
canonical representative, whereas the performance of operation two did change. In order to evaluate the 
operations together, the average amount of change between each operation at each $N$ value will be added together then 
divided by two in order to get the average amount of change for that particular $N$ across both operations. 
For example, with SJT the average change for $N=3\dots6$ will be asessed as follows. $N=3=(1 + 0.2)/2$, $N=4=(1+ 0.173913)/2, 
N=5=(1+0.134454), N=6=(1+0.109875)$. Each of these values represent the average amount of change between 
the two operations for each $N$ value in regards to the SJT Gray Code. The values for $N=3\dots6$ for
SJT are 0.600000, 0.5869565, 0.567227, 0.5549375. The values for $N=3 \dots 6$ for
Heaps are 0.900000, 1.000000, 1.092437, 1.100139.  The values for $N=3\dots6$ for Zaks 
are 0.900000, 1.173913, 1.302521, 1.345619. The values for $N=3\dots6$ for lexicographic are 
0.700000, 0.891304, 0.987395, 1.021558. Cleary SJT outperforms the other three Gray Codes by a substantial 
amount. This is because the difference between the amount of change for operation one 
across the four Gray Codes is much larger than it is for operation two. Therefore operation one has more 
weight when it comes to the average amount of change between the two operations than does operation two.
As $N$ increased form three to six, the average overall amount of change for SJT was 
0.577280 and the variance was 0.000403. Also notice that when $N=5$ the avearge amount of overall 
change is greater than when $N=6$ which suggests that the overall amount of change may decrease as 
$N$ approaches large values. As $N$ increased form three to six, the average overall amount of change for Heaps was 
1.023144 and the variance was 0.008810.
As $N$ increased form three to six, the average overall amount of change for Zaks was 
1.180513 and the variance was 0.040292. As $N$ increased form three to six, the average overall amount of change for lexicographic was 
0.900064 and the variance was  0.020830. In terms of overall change, SJT is the best, followed by lexicographic, followed by
Heaps, followed by Zaks. Notice how only SJT demonstrated a decerease in the overall average amount of change as $N$ went from five 
to six. Therefore there is only evidence that SJT will decrease in terms of overall amount of change as $N$ approaches 
large values.  