\subsection{Procedure}
Thus far, the problem has been introduced and the basic 
operations have been defined. Recall that there are two basic 
operations; insertion/deletion of a bar or swapping of two bars.
However, there has yet to be discussion regarding the four modified 
Gray Code algorithms used to generate the data. In the procedure section 
we look at the pseudo code for each of the algorithms and explain what 
each of the algorithms are doing. In each of these algorithms, only the root 
ladder is created. The algorithms can be modified by applying  the logic to 
a permutation then calling the enumeration algorithm which will generate $OptL\{\pi_{N}\}$.
This will allow the greedy approach to be applied for selecting the canonical representative.


%%section SJT
\subsubsection{Steinhaus-Johnson-Trotter}\pagebreak


\begin{algorithm}
  \begin{algorithmic}[1]
    \Function{modifiedSjt}{n, ladder, arr, direction}

      %%base case
      \If{$globalCount = n!$}
        \State return
      \EndIf

      %%initial case
      \If{$globalCount = 1$}
        \State $ladder \gets Idendity ladder$
      \EndIf

     
      \State $dir \gets direction[n-1]$

      %%swap the nth element n-1 times
      \For{$i \gets 0$,$i < n-1$, $i \gets i+1$}
        
        \State $arr[n-1] \gets arr[n-1]+1$
        \If{$dir = left$}
          \State add or remove a bar in column to the left
          \Else
            \State add or remove a bar in column to the right
        \EndIf
        \If{$ladder$ has degenerative triadic bar pattern from $basicOpOne$}
          \State swap the bars in the ladder
         \EndIf
        \State $globalCount \gets globalCount+1$

      \EndFor
      \State $direction[n-1] \gets !direction[n-1]$
      \State $HELPERSJT(n-1, ladder, arr, direction)$
      \State $MODIFIEDSJT(n,  ladder, arr, direction)$

    \EndFunction
  \end{algorithmic}
\end{algorithm}

%%helper algorithm
\begin{algorithm}
  \begin{algorithmic}[1]
    \Function{helpersjt}{n, ladder, arr, direction}
      \For{$i \gets n-1$, $i \geq 0$, $i\ gets i-1$}
        \If{$arr[i] < i$}
          \If{$dir[i] = LEFT$}
            \State add or remove a bar in column to the left 
          \Else
            \State add or remove a bar in column to the right
          \EndIf
          \State $arr[i]\gets arr[i+1]$
          \State return
        \Else 
          \State $arr[i] \gets 0$
          \State $direction[i] \gets !direction[i]$
        \EndIf
      \EndFor
      \EndFunction
  \end{algorithmic}
\end{algorithm}

The principles that act as the foundation for the modified SJT algorithm following. $N$ represents the length of the array that the ladder 
corresponds to. $arr$ represents the elements from $1$ to $N-1$ and is 
used to keep track of how many times a bar corresponding to one of their routes 
has been added or removed. $dir$ represents the current direction for 
each element from $1 \dots N$. The direction is switched when each element has had a 
bar added or removed from its route $N-1$ times going in one direction. 
Insert or remove a bar belonging to the route of the $Nth$ element going 
from left to right or right to left. Once the first or last column has had a bar 
belonging to the $Nth$ element added or removed, switch the direction for the $Nth$
element then remove or add a bar corresponding to the 
$N-Rth$ element where $R>1$. It is this stage in which $HELPERSJT$ is called. When a bar 
is added or removed from the $N-Rth$ element, the array at the position representing 
that element is increased by one. If the array at that index representing the element 
equals $N-R$ then reset it to zero and move onto the next element less than $N-R$. This step can be found 
on line(s) 3 and 10 in $HELPERSJT$. Once the $N-Rth$ element has had a bar added or removed from the first or last column, 
continue with $MODIFIEDSJT$ until $globalCount=n!$. The algorithm requires $O(1)$ time for the insertion or 
deletion of a bar. The procedure is done $n!$ times. Swapping two bars takes $O(1)$ time. The procedure is done 
$(n-1)!$ times.\pagebreak



%%Section: Heaps
\subsubsection{Heap's Algorithm}

\begin{algorithm}
  \begin{algorithmic}[1]
    \Function{modifiedHeaps}{Ladder, index, permuation}
      \If{$index=0$}
        \State return
      \EndIf
      \For{$i\gets 0, i<index, i\gets i+1$}
        
        $MODIFIEDHEAPS(Ladder, index-1)$
        \If{$index$ is even}
          \State $swap(perm[0], perm[index])$
        \Else 
          \State $swap(perm[i], perm[index])$
          
        \EndIf
        \State $Ladder \gets Root_{permuation}$
      \EndFor
      \If{$index > 1$}
        \State $MODIFIEDHEAPS(Ladder, index-1)$
      \EndIf
    \EndFunction
  \end{algorithmic}
\end{algorithm}
The principles that act as the foundation for the modified Heap's algorithm 
are the following. Initially begin with the Idendity ladder, when the index is 
even insert swap the element in $\pi$ at zero with the element and index-1. Then create the 
corresponding root ladder. Otherwise swap the value at the counter with the value at the 
index and create the corresponding root ladder. The time complexity for swapping two elements 
is $O(1)$. This is done $n!$ times. Add this to the cost of creating the root ladder which is 
$n^{3}$.\pagebreak



%%Section Zaks
\subsubsection{Zaks Algorithm}
%%main driver
\begin{algorithm}
  \begin{algorithmic}[1]
    \Function{modifiedZaks}{permutation, ladder, suffixVector, n, index}
      \If{$GlobalCount = n!$}
        \State return
      \EndIf
        \State $endIndx \gets n-1$
        \State $startIndx \gets n-suffixVector[index]$
        \State reverse $permutation's$ suffix starting from $startIndx$ to $endIndx$
        \State $ladder \gets Root_{permutation}$.
        \State $index \gets index+1$
        \State$MODIFIEDZAKS(permutation, ladder, suffixVector, n, index+1)$ 


    \EndFunction
  
  
  \end{algorithmic}

\end{algorithm}


%%helpe function
\begin{algorithm}
  \begin{algorithmic}[1]
    \Function{createsuffix}{suffixVector, n}
      \If{$n = 2$}
        \State append 2 to $suffixVector$
        \State return
      \EndIf

        $CREATESUFFIX(suffixVector, n-1)$
        
        $temp \gets$ suffixVector
        \For{$i \gets 0$, $i < n-1$, $i \gets i+1$}

          \State append $n$ to $suffixVector$
          \State append $temp$ to $suffixVector$
        \EndFor
    \EndFunction
  \end{algorithmic}
\end{algorithm}\pagebreak

The principles that act as the foundation for the modified Zaks 
algorithm are the following. First, the suffix vector is created. 
The suffix vector is created with the following recurrence relation.
\begin{center}
  $S_{2} = 2$\newline
  $S_{N} = (S_{N-1},N)^{n-1}N-1, n>2$\newline
\end{center}

\begin{example}
  \begin{center}


  $N=2$\newline
  $S_{2} = 2$\newline
  $(1,2), (2,1)$\newline

  $N=3$\newline
  $S_{3} = 23232$\newline 
  $(1,2,3), (1,3,2), (2,3,1), (2,1,3), (3,1,2), (3,2,1)$\newline
  \end{center}
\end{example}
The suffix vector represents the suffix of the permutation that is to 
be reversed. The procedure begings with the Idendity permuation of 
size $N$. If $N$ is $2$ then reverse the only two elements in $\pi$.
Else reverse the suffix of size $N-1$ $(N-1)!$ times, with $1$ being unmoved.
Then reverse the entire permutation, putting element $1$ at the end of the permuation.
Repeat these steps, now with element $2$ as the prefix, thus leaving element 
$2$ unmoved for the next round of $(N-1)!$ reversals. Repeat the process 
$N!$ which results in the reverse permuation.\par 
Once the suffix vector is created, on each call to the main function, use 
the current value of the suffix vector to indicate the size of the suffix to 
reverse, then reverse said suffix in the permutation and create the corresponding root ladder 
from the resulting permutation Reversing a suffix takes $O(N)$ time. This is done $N!$ times.
The cost of creating the root ladder is $O(N^{3})$.



%%Section Lex
\subsubsection{Lexicographic Algorithm}

\begin{algorithm}
  \begin{algorithmic}[1]
    \Function{modifiedlex}{perm, ladder, n}
      \If{$globalCount = n!$}
        \State return
      \EndIf
      \For{$i \gets n-1$, $i > 0$, $i \gets i-1$}
        \If{$perm[i-1]<perm[i]$}
          \State $maxMin \gets $ min value $>$ $perm[i-1]$ and to the right of 
          $perm[i-1]$
          
                                  
          \State $swap(perm[i-1], maxMin)$
          \State sort $perm$ from index $i$ to index $n-1$ in ascending order
          \State $ladder \gets Root_{perm}$
          \State $MODIFIEDLEX(perm, ladder, n)$
        \EndIf
      \EndFor

    \EndFunction
  \end{algorithmic}
\end{algorithm}

The principles founding the modified lexicographic algorithm are the following.
On each call to the algorithm, begin at the end of the permutation and 
find a decreasing subsequence going right to left. Once found, get the index 
of the lesser element. Then go to the right of the lesser element and find the 
smallest value greater than the lesser element. Swap this element with the lesser element. 
Sort the section of the array that is to the right of the original index of the 
lesser element before it was swapped. Once sorted, create the root ladder corresponding 
to this new permutation. The cost of sorting a suffix is $Nlog{N}$ and is done $N!$ times.
The cost of creating the root ladder is $N^{3}$.

\subsubsection{Algorithmic Analysis}
  It should be noted that only the modified SJT algorithm has been customized for ladder lotteries. The rest of the algorithms 
  require a permutation in order to generate the corresponding root ladder from the permutation. More work needs to be done in 
  order to create the proper modified Gray Code algorithms customized for ladder-lotteries. 
  Nonetheless, the above algorithms allow for analysis of the two basic operations which in turn can lead to determining which Gray Code is best 
  for enumerating the canonical representative. These results will be discussed in the next section.