\section{Coding Ladder Lotteries}
In \emph{Coding Ladder Lotteries}~\cite{A5}, written by Aiuchi, Yamanaka, Hirayama and Nishitani, the authors provide three methods to encode ladder lotteries as 
binary strings. Coding ladder lotteries as binary strings allows for a compact, computer readable, representation of them. Although we do not 
apply the encodings to the results of this thesis, we include them for completeness.
\subsection{Route Based Encoding}
The first method is termed \emph{route based encoding method} in 
which each route of an element in the permutation has a binary encoding. Let $RC(l)$ be the route encoding for 
some arbitrary ladder in $OptL\{\pi\}$. We encode $RC(l)$ by encoding the route of each $p_{i}$ in $\pi$. Let a $1$ 
for the encoding of $p_{i}$ indicate that $p_{i}$ is travelling left to right. Let a $0$ in the encoding of $p_{i}$ indicate 
that $p_{i}$ is travelling right to left. Then, append zero to $n-1$ $0s$ to the end of the encoding of $p_{i}$ so that the length 
of the encoding is $n-1$. Then, combine the encodings for each $p_{i}$ to produce $RC(l)$.   
For an example of the route encoding for the root ladder of $(3,2,5,4,1)$ refer to 
Figure~\ref{fig:route-encoding}; underlined $0s$ indicate appended $0s$. The number of bits needed for $RC(l)$ is $\mathcal{O}(n^{2})$.\par 
\begin{figure}[!htp]
    \begin{center}
        \begin{tikzpicture}
    
            %%draw the lines
            \draw(0, 0) to (0, 4);
                \node at(0, 4.3){3};
                \node at(0, -0.3){1};
            \draw(2, 0) to (2, 4);
                \node at (2, 4.3){2};
                \node at(2, -0.3){2};
            \draw(4, 0) to (4, 4);
                \node at (4, 4.3){5};
                \node at (4, -0.3){3};
            \draw(6, 0) to (6, 4);
                \node at (6, 4.3){4};
                \node at (6, -0.3){4};
            \draw(8, 0) to (8, 4);
                \node at (8, 4.3){1};
                \node at (8, -0.3){5};
    
            %%Draw the bars
            \draw(0, 2) to (2, 2);
            \draw(2, 1.5) to (4,1.5);
            \draw(0, 1) to (2, 1);

            \draw(4, 3) to (6, 3);
            \draw(6, 2.5) to (8, 2.5);
            \draw(4, 2) to (6, 2);
        \end{tikzpicture}
    \end{center}
   
 \caption{The route encoding for the given ladder lottery is \newline 11\underline{00}01\underline{00}11\underline{00}01\underline{00}0000}
 \label{fig:route-encoding}

\end{figure}

\subsection{Line Based Encoding}
The second method is termed \emph{line based encoding} which focuses 
on encoding the lines of the ladder lottery. Each line is represented 
as a sequence of endpoints of bars. Let $l$ be an optimal ladder lottery 
with $n$ lines and $b$ bars, then for some arbitrary line, $i$, there 
are zero or more right/left endpoints of bars that 
come into contact with $i$. Let $LC(i)$ denote the line based encoding for line $i$.
Let $1$ denote a left end point that 
comes into contact with line $i$ and let $0$ denote a right 
end point that comes into contact with line $i$. Finally, append a $0$
to line $i$ to denote the end of the encoding for that line. Then line $i$ can be 
encoded, from top to bottom, as a sequence of $1s$ and $0s$ that 
terminates in a $0$.  
Since each bar is encoded as two bits, and there are $n-1$ bits as terminating bits, 
one for each line in $l$, then the number of bits required is $n + 2b -1$, where $n$
is the number of lines and $b$ is the number of bars. Encoding and decoding can be 
done in $O(n+b)$ time. Clearly the line-based encoding 
trumps the route-based encoding in both time and space complexity. To see an example 
of the line based encoding, refer to Figure~\ref{Fig:LineBasedEncoding}. Terminating $0s$ 
are underlined.
\begin{figure}[ht]
    \centering
    \begin{tikzpicture}
        \draw(0, 0) to (0, 4);
            \draw(0, 3.5) to (2, 3.5);
            \draw(0, 1.5) to (2, 1.5);
        \draw(2, 0) to (2, 4);
            \draw(2, 2.5) to (4, 2.5);
        \draw(4, 0) to (4, 4);
            \draw(4, 3.5) to (6, 3.5);
        \draw(6, 0) to (6, 4);
        \node at(0, 4.2){$3$};
        \node at(2, 4.2){$2$};
        \node at(4, 4.2){$4$};
        \node at(6, 4.2){$1$};
       
        \node at(0, -.2){$1$};
        \node at(2, -.2){$2$};
        \node at(4, -.2){$3$};
        \node at(6, -.2){$4$};
    \end{tikzpicture}
    \caption{The line based encoding for the given ladder lottery is $11\underline{0}010\underline{0}10\underline{0}0$}
    \label{Fig:LineBasedEncoding}
\end{figure}

\subsection{Improved Line Based Encoding}
Although the line-based encoding is better than the route based 
encoding, it can still be further optimized. The authors provide 
three improvements to the line-based encoding.
\begin{enumerate}
\item The $nth$ line has only right endpoints attached to it, 
therefore it actually does not need to be encoded. Right endpoints 
are denoted as $0$ and left endpoints are encoded as $1$, therefore the number of right endpoints 
for line $n$ is equal to the number of $1s$ in $LC(i=n-1)$.
\item Given two bars,
$x,y$, let $l_{x}$ denote the left endpoint of bar $x$, let 
$l_{y}$ denote the left endpoint of bar $y$, let $r_{x}$ denote 
the right end point of bar $x$ and let $r_{y}$ denote the right 
end point of bar $y$. Let line $i$ be the line of $l_{x}$ and $l_{y}$
and let line $i+1$ be the line of $r_{x}$ and $r_{y}$. If there is not a $0$ between 
the two  $1s$ for $l_{x}$, $l_{y}$ in $LC_{i}$, it is implied that there is at least one $1$ between 
the two $0s$ for $r_{x}$, $r_{y}$ on $LC_{i+1}$. Hence, one of the $1s$ in $LC(i+1)$ can be omitted. 

% There are three possible cases for the placement of $x$ and $y$ in some arbitrary ladder from $OptL\{\pi\}$. 
% The first case is that there 
% is at least one other bar, $z$, with a right end point, $r_{z}$ between $l_{x}$
% and $l_{y}$ on line $i$. The second case is that there is at least one other bar 
% $z$, with a left end point, $l_{z}$, between $r_{x}$ and $r_{y}$ on line $i+1$. 
% The third case is that there is at least one bar, $z$, with a right end point, 
% $r_{z}$, between $l_{x}$ and $l_{y}$ on line $i$ and there is at least one other bar, 
% $z\prime$ with a left end point, $l_{z\prime}$, between $r_{x}$ and $r_{y}$ on line $i+1$~\cite{A5}. 
% If $l_{x}$ and $l_{y}$ on line $i$ have no $r_{z}$ between them, 
% then there must be at least one $l_{z\prime}$ between $r_{x}$ and $r_{y}$ on line $i+1$.
% Since a left endpoint is encoded as a $1$ and a right endpoint is encoded as a $0$, 
% a $1$ can be omitted for the encoding of line $i+1$ if $l_{x}$ and $l_{y}$ have no $r_{z}$
% between them on line $i$~\cite{A5}. That is to say, if there is not a $0$ between 
% the two  $1s$ for $l_{x}$, $l_{y}$ in $LC_{i}$, it is implied that there is at least one $1$ between 
% the two $0s$ for $r_{x}$, $r_{y}$ on $LC_{i+1}$. Hence, one of the $1s$ in $LC(i+1)$ can be omitted. 
% The line encoding with improvement two for the ladder in Figure~\ref{fig:line-encoding} is $11\underline{0}010\underline{0}00\underline{0}0$.
\item Improvement three is based off of saving some bits for right 
end points/$0s$ in $LC(i=n-1)$. Since line $n$ has no left end points,
then there must be some right endpoints between any two 
consecutive bars connecting lines $n-1$ and line $n$. Knowing this, then 
given two bars, $x$ and $y$ with $l_{x}$/$l_{y}$ on line 
$n-1$ and $r_{x}$/$r_{y}$ on line $n$, there must be at least 
one bar, $z$, with its $r_{z}$ between $l_{x}$ and $l_{y}$
on line $n-1$. Thus, for every $1$ in $LC(i=n-1)$ except the 
last $1$ a $0$ must immediately proceed any $1$. 
Since this $0$ is implied, it can be removed from $LC(i=n-1)$.
\end{enumerate} 
The line based encoding for the ladder in Figure~\ref{Fig:allthree} is $11\underline{0}101011\underline{0}0010101\underline{0}000$. 
The application of all three improvements can be done independently.\newline 
By applying improvement one, we get $11\underline{0}101011\underline{0}0010101\underline{0}$. Notice how the last three $0s$ were removed because they represented $LC(i=n)$.
By applying improvement two to we get $11\underline{0}10011\underline{0}001001\underline{0}$.
Notice how the second, and eighth $1$ were removed because they are implied by 
the successive $0s$. By applying improvement three to the result of improvement 
two we get $11\underline{0}10011\underline{0}00101\underline{0}$. Notice how the last $0$ 
was removed from improvement two. This is because the $0$ is implied in $LC(i=n-1)$
due to the configuration between of bars connecting lines $n-1$ and line $n$. The improved line based encoding 
for the ladder in Figure~\ref{Fig:allthree} 
is $11\underline{0}10011\underline{0}00101\underline{0}$.
\begin{figure}[h]
    \centering
     \begin{tikzpicture}
         \draw(0, 0) to (0, 4);
             \node at (0, 4.3){\small{$2$}};
             \draw(0, 3.5) to (1.5, 3.5);
             \draw(0, 2.5) to (1.5, 2.5);
         \draw(1.5, 0) to (1.5, 4);
             \node at (1.5, 4.3){\small{$1$}};
             \draw(1.5, 3) to (3, 3);
             \draw(1.5, 3.8) to (3, 3.8);
             \draw(1.5, 2) to (3, 2);
             \draw(1.5, 1) to (3, 1);
         \draw(3, 0) to (3, 4);
             \node at(3, 4.3){\small{$3$}};
             \draw(3, 2.5) to (4.5, 2.5);
             \draw(3, 1.5) to (4.5, 1.5);
             \draw(3, 0.5) to (4.5, 0.5);
         \draw(4.5, 0) to (4.5, 4);
             \node at(4.5, 4.3){\small{$4$}};
     \end{tikzpicture}
     \caption{A ladder used to illustrate all three improvements. The improved line based encoding is $11\underline{0}10011\underline{0}00101\underline{0}$}
     \label{Fig:allthree}
\end{figure}

We see that the improved line based encoding for the ladder in Figure~\ref{Fig:allthree} is 15 bits whereas the line based encoding for the 
same ladder is 21 bits. This is an improvement of roughly $28.5\%$.  