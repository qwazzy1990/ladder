\section{Enumeration, Counting, and Random Generation of \newline Ladder Lotteries}

In the paper, Enumeration, Counting, and Random Generation of Ladder Lotteries~\cite{A6}, written by Nakano and Yamanaka 
the authors consider the problem of enumeration, counting and 
random generation of ladder lotteries with $n$ lines and $b$ bars. 
It is important to note that the authors considered both optimal and 
non-optimal ladders for this paper. Nonetheless, the paper is still fruitful 
for its modelling of the problems and insights into ladder lotteries.
The authors use  the line-based encoding, $LC(l)$ for the representation of ladders 
that was discussed in the review of Coding Ladder Lotteries~\cite{A5}.

%%Section for enumeration
\subsection{Enumeration}
The authors denote a set of ladder lotteries with $n$ lines and 
$b$ bars as $S_{n,b}$. The problem is how to enumerate all the 
ladders in $S_{n,b}$. The authors use a \emph{forest structure}
to model the problem. A \emph{forest structure} is a set of trees 
such that each tree in the forest is disjoint union with every other 
tree in the forest. Consider $S_{n,b}$ to be a tree in a forest.
That is to say, a union disjoint subset of all ladders with $n$
lines and $b$ bars. Then $F_{n,b}$, or the forest of all $S_{n,b}$,
is the union of all disjoint trees of ladders with $n$ lines and $b$ bars. 
%For an example 
% of a forest for $F_{3,2}$ refer to Figure~\ref{fig:forest3,2}.\pagebreak %figure number.
% \begin{figure}[t]
%     \begin{center}
%     \begin{minipage}{.8\textwidth}
%         \begin{tikzpicture}
%             \draw(0, 0) to (0, 2);
%             \draw(0.5, 0) to (0.5, 2);
%                 \node at(0.25, -0.5){00};

%             %%branch
%             \draw[line width=0.5mm] (0.8, 1) to (1.3, 1);

%             \draw(1.5, 0) to (1.5, 2);
%                 \draw(2, 1.5) to (2.25, 1.5);
%             \draw(2, 0) to (2, 2);
%                 \node at (1.75, -0.5){0\underline{1}0};
    
%             %%branch
%             \draw[line width=0.5mm] (2.3, 1) to (2.8, 1);
            
%             \draw(3, 0) to (3, 2);
%             \draw(3.5, 0) to (3.5, 2);
%                  \draw(3.5, 1.5) to (3.75, 1.5);
%                  \draw(3.5, 1) to (3.75, 1);
%                 \node at (3.25, -0.5){01\underline{1}0};
            
%             \draw[line width=0.5mm] (4, 1) to (4.5, 1);

%             \draw(4.6, 0) to (4.6, 2);
%             \draw(5.1, 0) to (5.1, 2);
%             \draw(5.6, 0) to (5.6, 2);
%                  \draw(5.1, 1.5) to (5.35, 1.5);
%                  \draw(5.1, 1) to (5.35, 1);
%                 \node at (5.1, -0.5){011\underline{0}0};


%             \draw[line width=0.5mm](5.8, 1) to (6.3, 1);
%             \draw(6.5, 0) to (6.5, 2);
%             \draw(7, 0) to (7, 2);
%             \draw(7.5, 0) to (7.5, 2);
%                 \draw(7, 1.5) to (7.5, 1.5);
%                 \draw(7, 1) to (7.25, 1);
%                 \node at (7, -0.5){0110\underline{0}0};

%             \draw[line width=0.5mm](7.8, 1) to (8.3, 1);
%             \draw(8.6, 0) to (8.6, 2);
%             \draw(9.1, 0) to (9.1, 2);
%             \draw(9.6, 0) to (9.6, 2);
%                 \draw(9.1, 1.5) to (9.6, 1.5);
%                 \draw(9.1, 1) to (9.6, 1);
%                 \node at (9.1, -0.5){01100\underline{0}0};

%         \end{tikzpicture}
        
%     \end{minipage}
%     \end{center}
    
    
  

%     %%second tree
%     \begin{center}
%           \begin{minipage}{0.8\textwidth}
%             \begin{tikzpicture}
%                 \draw(0, 0) to (0, 2);
%                 \draw(0.5, 0) to (0.5, 2);
%                     \draw(0, 1) to (0.25, 1);
%                     \node at (0.25, -0.5){$100$};
                
%                 %%upper subtree
%                 \draw[line width=0.5mm](0.8, 1.5) to (1.3, 2);
%                     \draw(1.5, 1.5) to (1.5, 3.5);
%                     \draw(2, 1.5) to (2, 3.5);
%                         \draw(1.5, 3) to (2, 3);
%                         \node at (1.75, 1){$10\underline{0}0$};

%                     \draw[line width=0.5mm](2.3, 2.5) to (2.8, 2.5);
%                         \draw(3, 1.5) to (3, 3.5);
%                             \draw(3, 3) to (3.5,3);
%                             \draw(3.5, 2.5) to (3.75, 2.5);
%                         \draw(3.5, 1.5) to (3.5, 3.5);
%                         \node at (3.5, 1){$100\underline{1}0$};

%                     \draw[line width=0.5mm](4, 2.5) to (4.5, 2.5);
%                         \draw(4.7,  1.5) to (4.7, 3.5);
%                             \draw(4.7, 3) to (5.2, 3);
%                             \draw(5.2, 2.5) to (5.45, 2.5);
%                         \draw(5.2, 1.5) to (5.2, 3.5);
%                         \draw(5.7, 1.5) to (5.7, 3.5);
%                         \node at (5.2, 1){$1001\underline{0}0$};

%                     \draw[line width=0.5mm](6, 2.5) to (6.5, 2.5);
%                         \draw(6.8,  1.5) to (6.8, 3.5);
%                             \draw(6.8, 3) to (7.3, 3);
%                             \draw(7.3, 2.5) to (7.8, 2.5);
%                         \draw(7.3, 1.5) to (7.3, 3.5);
%                         \draw(7.8, 1.5) to (7.8, 3.5);
%                         \node at (7.3, 1){$10010\underline{0}0$};



%                 %%lower subtree
%                 \draw[line width = 0.5mm](0.8, 0.5) to (1.3, 0);
%                     \draw (1.5, -0.5) to (1.5, -2.5);
%                         \draw(1.5, -1.5) to (1.75, -1.5);
%                         \draw(2, -1) to (2.25, -1);
%                     \draw(2, -0.5) to (2, -2.5);
%                         \node at (1.5, -2.8){$10\underline{1}0$};
                    
%                     \draw[line width = 0.5mm](2.55, -1.5) to (3.05, -1.5);

%                      \draw (3.3, -0.5) to (3.3, -2.5);
%                         \draw(3.3, -1.5) to (3.8, -1.5);
%                         \draw(3.8, -1) to (4.05, -1);
%                     \draw(3.8, -0.5) to (3.8, -2.5);
%                         \node at (3.6, -2.8){$101\underline{0}0$};

%                      \draw[line width = 0.5mm](4.25, -1.5) to (4.75, -1.5);

%                      \draw (5, -0.5) to (5, -2.5);
%                         \draw(5.5, -1) to (5.75, -1);
%                         \draw(5, -1.5) to (5.5, -1.5);
%                     \draw(5.5, -0.5) to (5.5, -2.5);
%                     \draw(6, -0.5) to (6, -2.5);
%                         \node at (5.5, -2.8){$1010\underline{0}0$};
                    

%                     \draw[line width = 0.5mm](6.3, -1.5) to (6.8, -1.5);
%                     \draw (7.1, -0.5) to (7.1, -2.5);
%                         \draw(7.1, -1.5) to (7.6, -1.5);
%                         \draw(7.6, -1) to (8.1, -1);
%                     \draw(7.6, -0.5) to (7.6, -2.5);
%                     \draw(8.1, -0.5) to (8.1, -2.5);
%                         \node at (7.6, -2.8){$10100\underline{0}0$};
                    

                    
%             \end{tikzpicture}
            
%         \end{minipage}
%     \end{center}


%     \begin{center}
%         \begin{minipage}{0.8\textwidth}
%             \begin{tikzpicture}%% start pocture
            
            
%                 \draw(0, 0) to (0, 2);
%                     \draw(0, 1.5) to (0.25, 1.5);
%                     \draw(0, 1) to (0.25, 1);
%                 \draw(0.5, 0) to (0.5, 2);
%                 \node at (0.25, -0.5){$1100$};
            
            
%                 \draw[line width=0.5mm] (0.8, 1) to (1.3, 1);
%                     \draw(1.5, 0) to (1.5, 2);
%                         \draw(1.5, 1.5) to (2, 1.5);
%                         \draw(1.5, 1) to (1.75, 1);
%                     \draw(2, 0) to (2, 2);
%                     \node at (1.75, -0.5){$110\underline{0}0$};
            
        
%                 \draw[line width=0.5mm] (2.3, 1) to (2.8, 1);
%                     \draw(3, 0) to (3, 2);
%                        \draw(3, 1.5) to (3.5, 1.5);
%                        \draw(3, 1) to (3.5, 1);
%                    \draw(3.5, 0) to (3.5, 2);
%                     \node at (3.25, -0.5){$1100\underline{0}0$};
            
            
%                 \draw[line width=0.5mm] (3.8, 1) to (4.3, 1);
%                     \draw(4.5, 0) to (4.5, 2);
%                           \draw(4.5, 1.5) to (5, 1.5);
%                           \draw(4.5, 1) to (5, 1);
%                       \draw(5, 0) to (5, 2);
%                       \draw(5.5, 0) to (5.5, 2);
%                           \node at (5, -0.5){$11000\underline{0}0$};
               
%             \end{tikzpicture}%%end picture
%         \end{minipage}
%     \end{center}
%     \caption{The forest, $F_{3,2}$ where $3$ is the number of lines and $2$ is the number of bars. All ladders with $3$ lines and $2$ bars are leaf nodes of one of three trees $S_{3,2}$.
%     The underlined bits are the inserted second last bit from the parent's line-encoding resulting in the child's line encoding}
%     \label{fig:forest3,2}
% \end{figure}

%%end enumeratiom section

%%section for counting
\subsection{Counting}
The authors provide a method and algorithm to count all ladders 
with $n$ lines and $b$ bars. The counting algorithm 
works by dividing ladders into four types of sub-ladders.
For sub-ladder, $r$, its type is a tuple $t(n,h,p,q)$ where 
$n$ is the number of lines, $h$ is the number of half bars, 
$p$ is the number of unmatched end-points on line $n-1$ and 
$q$ is the number of unmatched end-points on line $n$. From this 
type, the authors are able to count all ladders with $n$ lines and $b$ bars. 
% \subsubsection{ $h < p+q$ or $n<2$}
% There are zero ladders because it is impossible for the 
% root sub-ladder to have less than two lines. It is also 
% impossible for the number of half bars, $h$, to 
% be less than the number of detached left end points 
% on line $n-1$ plus the number of detached end points on 
% line $n$.

% %%\subsubsubsection{Case 2: $n=2$ and $h=p$ and $q=0$}
% \subsubsection{$n=2$ and $h=p$ and $q=0$}
% There is only one ladder because the number of half bars 
% on the last line is 0 since $q=0$. Therefore all half bars are on the 
% $n-1th$ line of the sub-ladder. This is known because 
% $h=p$ which means the number of half bars is the same as 
% the number of unmatched bars on line $n-1$. Hence, the unmatched 
% half bars on the $n-1th$ line must be connected to the $n$ 
% line. Once these are all matched the ladder will be complete. 
% Thus, there is only one ladder for this case.

% %%\subsubsubsection{Case 3: $(n \geq 3$ or $h>p)$ and $q=0$}
% \subsubsection{$(n \geq 3$ or $h>p)$ and $q=0$}
% If this is the case, then there are no endpoints attached to 
% line $n$, but the number of half bars is greater than the 
% number of endpoints attached to line $n-1$, which means there is 
% some line(s) $n-t$, $t>2$ that have end points attached to them.
% Let $r$ be a sub-ladder of type $r=t(n,h,p,q)$
% with the the above values for $n,h,p,q$. In order to count the number of ladders of type 
% $t(n\geq3, h>p, q=0)$ the authors demonstrate an injection $|t(n\geq3, h>p, q=0)|=|t(n-1,h,0,p)| + |t(n,h-1,p+1,q)|$.\cite{A6}
% Let $P(r)$ be $r$ with the removal of $r's$ second last bit in $LC(r)$; i.e. the parent of 
% $r$.  The $LC(r)$ must have a $0$ for the second last bit. This $0$ designates either the 
% end of line $n-1$ or a right endpoint of a bar attached to line $n-1$. 
% If the second last bit in $LC(r)$ is the right end point of some 
% bar, then $P(r)=t(n,h-1,p+1,q)$. This is because the $n-1th$ bar 
% has a right end point that must be connected to some left  
% endpoint at line $n-2$. Since the removal sequence of the second 
% last bit ensures that there cannot be a right end-point detached 
% from a left end-point. Only left end-points can be detached 
% from right end-points~\cite{A6}. However, if the second last bit 
% of $LC(r)$ designates the end of line $n-1$, then $P(r)=t(n-1,h,0,p)$. 
% This is because the removal of the second last bit 
% is the removal of the end of line $n-1$ in $r$. Thus, 
% line $n$ must be empty in $r$ since the last bit in $LC(r)$
% designated the end of line $n$. Thus, if line $n$ is empty 
% and the end point of line $n-1$ has been removed from $LC(r)$, 
% resulting in $P(LC(r))$, the last bit in $P(LC(r))$ must be 
% the end of line $n-1$ in $r$ resulting in a pre-ladder with one 
% less line than $r$.\par  

% \subsubsection{ $h\geq p+q$ and $q>0$}
% %%\subsubsubsection{Case 4: $h\geq p+q$ and $q>0$}
% Let $r$ be a pre-ladder of type $t(n,h,p,q)$. The authors 
% demonstrate $|t(n,h\geq p+q,q>0)|=|t(n,h-1,p+1,q)|+|t(n,h-1,p,q-1)|$.\cite{A6} 
% The second last bit of $LC(r)$ is either a $0$ 
% or a $1$. If it is a $0$ then it represents a 
% right end point attached to line $n$. Thus, 
% removing it to get $P(LC(r))$ is in effect 
% detaching a right end point from some left end point 
% on line $n-1$. Therefore, the parent, $P(R)$ is 
% of type $t(n,h-1,p+1,q)$. Seeing as in the parent, 
% there is now a left end point detached from its right 
% end point in $r$. However, if the second last bit 
% of $LC(r)$ is a $1$, then this indicates the left 
% half of a bar on line $n$. But since there is no 
% bar $n+1$, this left end point must be detached. 
% Therefore, by removing this $1$ in $LC(r)$ results 
% in a parent with one less detached end point on line $n$.
% Thus $P(R)$ is of type $t(n,h-1,p,q-1)$.
% \subsection{Random Generation}
% The random generation of ladder lotteries with $n$ lines and
% $b$ bars is done by the recurrence relations in the counting 
% and enumerating sections. The goal is to produce 
% some $L$ of type $t(n,2b,0,0)$ where the number of half 
% bars equals the total $2(b)$ and there are no detached 
% end points on lines $n-1$ and $n$. This implies that there 
% are no detached endpoints on any line $n-t$ where $t\geq2$
% because the removal sequence from the $LC(pre-ladder)$
% ensures that any line before $n-1$ has no detached endpoints. Thus, 
% if $L$ is of type $t(n,2b,0,0)$ it is no longer a pre-ladder 
% but a complete ladder with $n$ lines and $b$ bars~\cite{A6}.\par 
% The authors use an algorithm to generate a random integer, $x$,
% in the range of $[1 \dots |t(n,h,p,q)|]$. where $t(n,h,p,q)$ corresponds to some 
% parent type of ladder. $t(n1,h1,p1,q1)$ corresponds to one 
% child type of $t(n,h,p,q)$ and $t(n2,h2,p2,q2)$ corresponds 
% to the other child type. If $x\leq|t(n1,h1,p1,q1)|$ then generate 
% a pre-ladder of type $t(n1,h1,p1,q1)$ else generate a pre-ladder 
% of type $t(n2,h2,p2,q2)$~\cite{A6}. Continue until there is type $t(n,2b,0,0)$
% which corresponds to a complete ladder lottery with $n$ lines and $b$ bars.