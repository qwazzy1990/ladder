
In Chapter 2 we will provide a more comprehensive analysis of the existing research surrounding ladder lotteries. 
We will also go through the foundational concepts, algorithms and proofs that act as the backbone for the 
results in Chapter 3 and Chapter 4. We have already defined $\pi$ in the introduction and we have already 
discussed a ladder lottery as a concept. 
Let $n$ be the number of elements in $\pi$. Let the \emph{ladder data structure}, or \emph{ladder} for short, be a two dimensional array 
with$0 \leq k \leq {n\choose 2}$ rows and $n-1$ columns. The number of rows in ladder is zero if $\pi$ is in ascending order.
The number of rows in ladder is ${n\choose 2}$ if $\pi$ is in descending order. The number of columns is always $n-1$ 
seeing as the columns represent gaps between lines. Thus, given $n$ lines, there are $n-1$ columns in ladder. 
Let $\tau(p_{i},p_{j})$ be an adjacent transposition of an adjacent inversion $(p_{i},p_{j})\in \pi$.
Throughout the thesis a number of algorithms are presented. Many of these algorithms use the following auxiliary functions: 
\begin{enumerate}
    \item {\sc Print(x)}: Prints x.
    \item {\sc Swap(x,y)}: Swaps x and y.
    \item {\sc Sort(x)}: Sorts x in ascending order.
    \item {\sc Max(x)}: Returns the maximum element in x.
    \item {\sc Min(x)}: Returns the minimum element in x.
\end{enumerate}

The study of ladder lotteries as mathematical objects began in 2010, in the paper
Efficient Enumeration of Ladder Lotteries and its Application, written by Matsui, Nakada, Nakano Uehara and Yamanaka~\cite{A1}. 
In this paper the authors present the first algorithm for generating $OptL\{\pi\}$ for some  
arbitrary permutation $\pi$. Since this paper emerged, there have been 
a number of other papers written about ladder lotteries.
These papers include The Ladder Lottery Realization Problem,
Optimal Reconfiguration of Optimal Ladder Lotteries, 
Efficient Enumeration of all Ladder Lotteries with K Bars,
Coding Ladder Lotteries and
Enumeration, Counting, and Random Generation of Ladder Lotteries. It is in these papers that the aforementioned 
problems related to ladder lotteries are solved. Thus, a comprehensive analysis of these papers 
is required for getting the full breadth of the literature surrounding ladder lotteries. 
This thesis is also heavily influenced by Efficient Enumeration of Ladder Lotteries and its Application. 
Some of the foundational concepts, algorithms and proofs will be 
discussed in the subsection pertaining to Efficient Enumeration of Ladder Lotteries and its Application seeing as 
the background information for this thesis 
is intricately related to the findings in the paper Efficient Enumeration of Ladder Lotteries and its Application.\par 

