
In this chapter we will provide a more comprehensive analysis of the existing research surrounding ladder lotteries. Let {\sc x} be 
a ladder lottery or permutation. 
Throughout this thesis a number of algorithms are presented. Many of these algorithms use the following auxiliary functions: 
\begin{enumerate}
    \item {\sc Print(x)}: Prints {\sc x}.
    \item {\sc Swap(x,y)}: Swaps {\sc x, y}.
    \item {\sc Sort(x)}: Sorts {\sc x} in ascending order.
    \item {\sc Sorted(x)}: Returns true if {\sc x} is sorted in ascending order, else returns false.
    \item {\sc Max(x)}: Returns the maximum element in {\sc x}.
    \item {\sc Min(x)}: Returns the minimum element in {\sc x}.
\end{enumerate}

The study of ladder lotteries as mathematical objects began in 2010, in the paper
Efficient Enumeration of Ladder Lotteries and its Application, written by Matsui, Nakada, Nakano Uehara and Yamanaka~\cite{A1}. 
In this paper the authors present the first algorithm for generating $OptL\{\pi\}$ for some  
arbitrary permutation $\pi$. Since this paper emerged, there have been 
a number of other papers written about ladder lotteries.
These papers include The Ladder Lottery Realization Problem,
Optimal Reconfiguration of Optimal Ladder Lotteries, 
Efficient Enumeration of all Ladder Lotteries with K Bars,
Coding Ladder Lotteries and
Enumeration, Counting, and Random Generation of Ladder Lotteries.
This thesis is also heavily influenced by Efficient Enumeration of Ladder Lotteries and its Application. Throughout Chapter 2, 
we elaborate on the aforementioned papers pertaining to ladder lotteries.

