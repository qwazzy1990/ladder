
In summary, we have defined the canonical ladder, designed a representation of the canonical ladder along with equations to 
calculate the coordinates of the bars in the representation, implemented an algorithm to create the canonical ladder, 
implemented a constant amortized algorithm to list $L_{n}$ in Gray code order by adding or removing a bar, 
designed a constant amortized algorithm to list $L_{n,k}$ in Gray code order, and designed a 
constant amortized algorithm to list $L_{n}$ by listing $L_{n, 0} \dots L_{n, k} \dots L_{n, {n \choose 2}}$. 
The algorithms can be used to list sorting network arrangements that correspond to the canonical ladders.\par 
We are left with the following open problems:
\begin{enumerate}
    \item Proving Conjecture~\ref{Conjecture:ReverseOrdering}
    \item Design and implement {\sc ReverseWalsh}
    \item Implementing {\sc ListLNKReverse}, assuming we can implement {\sc ReverseWalsh}
    \item Implement {\sc ListLnByKBars}, assuming we can implement {\sc ListLNKReverse}.
    \item Continuing to work on the counting problem for the number of ladders in $OptL\{\pi}\}$, specifically 
    when $\pi=(n,n-1, \dots, 2,1)$.  
\end{enumerate}