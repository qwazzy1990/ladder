\subsection{Optimal Reconfiguration of Optimal Ladder Lotteries}
In \textbf{Optimal Reconfiguration of Optimal Ladder Lotteries},
the authors provide a polyminomial solution to the 
minimal reconfiguration problem. The problem states that given 
two ladder is $OptL\{\pi\}$, $L_{i}$ and  $L_{m}$, what is the minimal number of 
local swap operations to perform that will transition from $L_{i}$ to $L_{m}$ \cite{A2}.
The authors do so based on the local swap operations previously 
discuessed along with some other concepts. The first of these concepts 
is termed the \emph{reverse triple}. Basically, a reverse triple is a relation
between three bars, $x,y,z$ in two arbitrary ladders, $L_{i}, L_{m}$, such that if $x,y,x$
are right rotated in one of the ladders, then they are left rotated in the other. 
The second of the concepts is the \emph{improving triple}. The improving triple is 
essentially a bar that can be left/right rotated such that the 
result of the rotation the bar removes a reverse triple between two arbitrary
ladders $L_{i}$ and $L_{m}$ \cite{A2}. The solution to transition from 
$L_{i}$ to $L_{m}$ with the minimal length reconfiguration sequence 
is achieved by applying improving triple to the reverse triples between 
$L_{i}$ and $L_{m}$. That is to say, the length of the reconfiguration sequence 
is equal to the number of reverse triples between $L_{i}$ and  $L_{m}$ \cite{A2}.\par
The second contribution of this paper is that it provides a closed form 
upper bound for the minimal length reconfiguration sequence for any permutation 
of size $N$. That is to say, given any permutation, $\pi$, of size $N$ what is the maximum 
length of a minimal reconfiguration sequence between two ladders in $OptL\{\pi\}$.
The authors prove that it is $OptL\{\pi_{N, N-1, \dots, 1}\}$ that contains the 
upper bound for the minimal length reconfiguration sequence between two ladders $L_{i}$ and 
$L_{m}$ \cite{A2}. Moreover, it is only the root ladder and terminating ladder in 
$OptL\{\pi_{N, N-1, \dots, 1}\}$ whose minimal reconfiguration sequence is equal to 
the upper bound. That upper bound is $N{(N-1) \choose 2}$. This is because 
the number of reverse triples between the root ladder and the terminating ladder 
in $OptL\{\pi_{N, N-1, \dots, 1}\}$ is equal to $N{(N-1) \choose 2}$. Thus, in 
order to reconfigure the root to the terminating ladder, or vice versa, each 
reverse triple between them must be improved.

