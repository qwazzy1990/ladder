\section{Permutations}
\subsection{Introduction}
Ladder lotteries and permutations are intricately related to each other. Each $\pi$ has an $OptL\{\pi\}$ such that 
each ladder form $OptL\{\pi\}$ sorts $\pi$. The so called Listing Problem is one of the problems addressed in this thesis.
In brief, this problem is about how to list all $N!$ ladders efficiently. The research for this problem is highly influenced 
by permutation enumeration research. Knuth describes a number of 
permutation enumeration algorithms in \textbf{The Art of Computer Programming} \cite{A18}. Since this book, many algorithms for 
enumerating $N!$ permutations have been created. During my research for the Listing Problem, I investigated five of these 
enumeration algorithms. They are the lexicographic algorithm, Heap's algorithm, Zak's algorithm, Steinhaus-Johnston Trotter's algorithm 
and Effler Rusky's algorithm. Each of these algorithms will be examined in detail.

\subsection{Lexicographic Algoritms}

%%
\subsubsection{Basic Lexicographic Algorithm}
The lexicographic algorithm enumerates permutations in order from smallest to 
largest. The lexicographic order for $S_{3}$ is $(1,2,3),(1,3,2),(2,1,3),(2,3,1),(3,1,2),(3,2,1)$.
This is the same way the words in a dictionary are ordered in the sense that $a < b < c \dots y < z$.
There are many algorithms for enumerating permutations lexiographically \cite{A19, A20, A21, A22, A23}. 
Several will be discussed in this thesis. The first of which is the basic lexicographic ordering algorithm.
Please refer to Alg.\ref{Alg:BasicLex} to see the basic lexicographic listing algorithm.\pagebreak
\begin{algorithm}
    \begin{algorithmic}[1]
        \Function{Basic Lex}{$\pi$, $N$}
            \For{$i \gets N, i >= 1, I \gets i-1$}
                \If{$\pi_{i} > \pi_{i-1}$}
                    \State $v \gets Min(\pi$ \textbackslash $(\pi_{0},\pi_{1}, \dots \pi_{i-1})) | v > \pi_{i-1}$.
                    \State $Swap(\pi_{i-1}, v)$ in $\pi$.
                    \State $Sort(\pi_{i},\pi_{i+1} \dots \pi_{N})$
                    \State break
                   
                \EndIf
            \EndFor
            \State $BaiscLex(\pi, N)$

        \EndFunction
        
    \end{algorithmic}
    \caption{The Basic Lexicographic Algorithm}
    \label{Alg:BasicLex}
\end{algorithm}
This algorithm works by beginning with the identity permutation of order $N$. Then going right to left, 
it finds an increasing substring of size two. Once found, the algorithm finds the smallest value greater than the 
value at index $i-1$ in $\pi$ to the right of index $i-1$; let this value be known as $V$. 
Then the algorithm swaps $V$ $\pi_{i-1}$ in $\pi$. The algorithm sorts $\pi$ from index $i$ to index $N$ and a recursive call is made.
The algorithm terminates when $\pi$ is in decending order.\par
\begin{lemma}
    The time complexity for the basic lexicographic algorithm is $O(N!)*N^{2}log(N)$.
\end{lemma}
\begin{proof}
    The algorithm generates all $N!$ permutations which accounts for the $N!$ factor. The for loop runs between 
    $0$ to $N$ times per function call which accounts for the $N$ factors. Lastly, the right portion of the permutation 
    needs to be sorted on every function call. Sorting is done in $Nlog_{N}$ time. Thus, $N*Nlog_{N}$ is simplified to 
    $N^{2}log_{N}$. Thus, we get $O(N!*N^{2}log_{N})$. End of proof.
\end{proof}


%%p2
\subsubsection{Alternating Lexicographic Algorithm}
In their paper, \emph{Generating Alternating Permutations Lexicographically}, the authors provide an algorithm 
for generating permutations in lexicographic ordering such that the permutations form a zig-zag pattern. 
The zig-zag pattern is formally deifned as follows, given $\pi$ of order $N$:  If $N = 2K$ then 
zig-zag = $\pi_{1} < \pi_{2} > \pi_{3} < \pi_{4} \dots \pi_{N-1} < \pi_{N}$. 
If $N= 2K+1$ then zig-zag = $\pi_{1} < \pi_{2} > \pi_{3} < \pi_{4} \dots \pi_{N-1} > \pi_{N}$ \cite{A19}.
Please refer to Alg.\ref{Alg:AltLex} to see the Alternating Lexicographic Algorithm.


\begin{algorithm}
    \begin{algorithmic}[1]
        \Function{AlternatingLex}{$m,val,level: int$, $s=\{1 \dots N\}$, $\pi$}
            \State $\pi_{level} \gets val$
            \State $s \gets s-\{val\}$
            \If{$m = 1$}
                \State $print(\pi)$
                \State return
            \EndIf
            \State $t \gets \{\}$
            \If{$level=2k+1$}
                \State $t \gets \{x \in s | x < s_{max}$ and $x < val\}$
            \Else 
                \State $t \gets \{x \in s | x > s_{min}$ and $x > val\}$
            \EndIf
            \For{$x \in t$}
                \State $AlternatingLex(m-1, x, level+1, s - \{x\}, \pi)$
            \EndFor
        \EndFunction
        
    \end{algorithmic}
    \caption{Alternating Lexicographic Enumeration Algorithm}
    \label{Alg:AltLex}
\end{algorithm}

The initial values for the funtion are $m=N$, $val=1$, $level=0$ and $s = \{1 \dots N\}$ and $\pi=()$.
On each function call $val$ is inserted into $\pi$ at $level$. Then $val$ is removed from $s$.
If $level$ is odd then the set $t$ gets every value from $s$ less than the max value in $s$ and 
less than $val$. If  $level$ is even then $t$ gets every value from $s$ greater than  the min value 
in $S$ and greater than $Val$. Then, for each $x$ in $T$, the function makes a recursive call 
with $m$ equal to $m-1$, $k$ equal to $x$, $level$ equal to $level+1$ and $s$ equal to $s$ without 
element $x$. To see the permutations of order $5$ generated by the alternating lexicographic algorithm, 
please refer to Table \ref{Table:AltPerms}.

\begin{center}
\begin{table}
\begin{tabular}{ |p{3cm}||p{3cm}|p{3cm}|p{3cm}|  }
 \hline
 \multicolumn{4}{|c|}{Alternating Permutations of order $5$} \\
 \hline
 \hline
 13254   & 14253    &14352&  15243\\\hline
 15342&  23154  & 24153   &24351\\\hline
 25143 &25341 & 34152&  34251\\\hline
 35142    &35241 & 45132&  45231\\
 \hline
\end{tabular}
\label{Table:AltPerms}
\end{table}
\end{center}

The authors state that the algorithm is \emph{constant average time} which means the total amount of
computation done in generating all the objects, divided by the number of objects, is
bounded by a constant. On the average and up to a constant factor no algorithm can
run faster \cite{A19}. The constant refers to the number of function calls before the algorithm terminates. It is defined as follows. 
Let $AL$ denote the recurrence relation for the Alternating Lexicographic algorithm, let $k$ denote the 
first element in $\pi$ and let $N$ denote the number of elements in the set. 
If $N=1$ and $K=1$ then $AL(N, K)=0$. If $N=N$ and $K=N-1$ then $AL(N,K)= 1 + AL(N-1,1)$ For all other cases, 
then $AL(N, K) = AL(N, K+1) + A(N-1, N-K)$ \cite{A19}.

% \subsubsection{Parallelized Lexicographic Algorithms}
% The lexicographic listing algorithm can be improved by parallelizing the process. There are a number of algorithms for 
% parallelizing the lexicographic listing problem. These algorithms are found in the following papers \cite{A21, A22, A23, A24}.
% Rather than go through these papers one by one, a more general approach to parallelized solutions of the 
% lexicographic listing problem will be provided.\par 
% The general solution to the problem is to partition the $N$ elements into $M$ procecssors, then have each processor 
% list $(N/M)!$ subsets of elements. Though these algorithms are interesting, the analysis of these algorithms was not 
% pertinent to the research in this thesis, seeing as the Listing Problem runs on a single processor.

\subsection{Heap's Algorithm}
Heap's algorithm was developed by B.R. Heap in 1963 \cite{A24}. 
The algorithm is based on rotating elements in an array such that the $Nth$ position 
is occupued by the $Nth$ element for all permutations of the $(N-1)$ objects. 
Then the $Nth$ element is swapped with one of the elements of in the first $(N-1)$ positions.
The process repeats itself until all of the $N$ elements have occupied the $Nth$ positions. 
To see Heap's algorithm please refer to Alg.\ref{Alg:Heaps}.

\begin{algorithm}
    \begin{algorithmic}[1]
        \Function{Heaps}{$\pi$, $N$}
            \If{$N=1$}
                \State $Print(\pi)$
            \Else 
                \For{$i \gets 1, i \leq N, i \gets i+1$}
                    \State $Heaps(\pi, N-1)$
                    \If{$N = 2k+1$}
                        \State $Swap(\pi_{1}, \pi_{N})$
                    \Else 
                        \State $Swap(\pi_{i}, \pi_{N})$
                    \EndIf
                \EndFor
            \EndIf
        \EndFunction
        
    \end{algorithmic}
    \caption{Heaps Algorithm for Generating all $N!$ Permutations}
    \label{Alg:Heaps}
\end{algorithm}

\begin{lemma}
    The time complexity of Heap's algorithm is $O(N!)$
\end{lemma}
\begin{proof}
    For each value of $N=[1 \dots N]$, the algorithm makes $N$ recursive calls, each of which produces all $(N-1)!$ permutations of 
    order $(N-1)$. Thus, you get $N(N-1)!=N!$ which is $O(N!)$. End of proof.
\end{proof}

\subsection{Zak's Algorithm}
Zak's Algorithm was first written about by Shmuel Zak's in the paper \textbf{A New Algorithm For
Generation Of Permutations} \cite{A31}.The algorithm is based on reversing suffixes in a permutation 
of different sizes until all $N!$ permutations have been listed. The algorithm makes 
use of a \emph{suffix vector} which is a vector for holding all the suffix sizes to be reversed. 
Since there are $N!$ permutations, there are $N!$ non-unique suffix sizes held in the suffix vector.
The recurrence relation for the suffix sizes is as follows. Let $S(N)$ denote the suffix  of size $N$. 
Then we get the following recurrence relation for $S(N)$. If $N=2$ then $S(N)=2$, else 
$S(N)=(S(N-1)N)^{N-1}S(N-1)$. Let $V(N)$ denote the suffix vector of size $N$. Then if $N$ equals 
two, append $2$ to $V(N)$, else append $V(N-1)$ followed by $N$ $N-1$ times to $V(N)$ \cite{A31}. On the $Nth$ time 
append $V(N-1)$. Once the suffix vector has been crated, then for each suffix size in the suffix vector, reverse 
the suffix of that size in $\pi$. To see Zak's Algorithm for creating the suffix vector please refer to 
Alg.\ref{Alg:Zak's}

\begin{algorithm}
    
    \begin{algorithmic}[1]
        \Function{$Suffix Vector$}{$vector$, $N$}
        \If{$N=2$}
           \State append $2$ to $vector$
        
        \Else 
            \For{$i \gets 1, i < N, i \gets i+1$}
                \State append $SuffixVector(vector, N-1)$ to $vector$
                \State append $N$ to $vector$
            \EndFor
            \State append $SuffixVector(vector, N-1)$ to $vector$
        \EndIf
        \EndFunction
\end{algorithmic}
    \caption{Creating the suffix vector}
    \label{Alg:Zaks}
\end{algorithm}


\begin{lemma}
    The time complexity of Suffix Vector is $O(N!)$
\end{lemma}
\begin{proof}
    For each value of $N=[1 \dots N]$, the algorithm makes $N$ recursive calls, each of which produces all $(N-1)!$ permutations of 
    order $(N-1)$. Thus, you get $N(N-1)!=N!$ which is $O(N!)$. End of proof.
\end{proof}


\begin{lemma}
    The time complexity to create all $N!$ is $O(N!(N))$
\end{lemma}
\begin{proof}
    Reversing a suffix is done in $O(N)$ time. The reversal happens $N!$ times. Therefore $O(N(N!))$. End of proof.  
\end{proof}



%Paragraph SJT
\subsection{Steinhaus-Johnston Trotter Algorithm}
The Steinhaus-Johnston Trotter algorithm generates $S_{N}$ by performing adjacent swap operations 
on the permutation resulting in the next permutation. Thus, each permutation in $S_{N}$ differs 
from it predecessor by a single swap operation \cite{A25}. This makes the SJT algorithm a very efficient 
algorithm for listing $S_{N}$. Let an \emph{even permutation} be defined as a permutation 
with an even number of inversion. Let an \emph{odd permutation} be defined as a 
permutation with an odd number of inversions. The $Nth$ element is inserted into all 
positions of $\pi$ of order $N-1$ in decending order if $\pi$ of order $N-1$ is an even permutation.
The $Nth$ element is inserted into all positions of $\pi$ of order $N-1$ in ascending order if 
$\pi$ of order $N-1$ is an odd permutation \cite{A25}. For $\pi$ of order $1$ we have $\pi=(1)$. Since 
there are no inversions in $\pi=(1)$ it is even. Now insert $2$ in all positions in $\pi=(1)$
in decending order. Thus we get $(1,2)$ followed by $(2,1)$. Since $(1,2)$ is 
an even permutation, insert $3$ into all positions in decending order resulting in $(1,2,3)$,
$(1,3,2)$ and $(3,1,2)$. Since $(2,1)$ is an odd permutation, insert $3$ 
into all permutations in ascending order resulting in $(3,2,1)$, $(2,3,1)$ and $(2,1,3)$. 
To see the Steinhaus-Johnston Trotter Algorithm please refer to Alg.\ref{Alg:SJT}

\begin{algorithm}
    \begin{algorithmic}[1]
        \Function{SJT}{$\pi$, $currentElement$, $N$, $direction=[]$}
            \If{$currentElement > N$}
                \State $print(\pi)$
                \State return
            \EndIf
            \For{$i \gets 1, i < currentElement, i \gets i+1$}
                \State $SJT(\pi, currentElement+1, N)$
                \State $index \gets $ index of $currentElement$ in $\pi$
                \If{$direction[currentElement]=true$}
                    \State $Swap(\pi_{index}, \pi_{index-1})$
                \Else 
                    \State $Swap(\pi_{index}, \pi_{index+1})$
                \EndIf
            \EndFor
            \State $SJT(\pi, currentElement+1, N, direction)$
            \State $direction[currentElement] \gets !(direction[currentElement])$
        \EndFunction
        
    \end{algorithmic}
    \caption{SJT Algorithm for listing $S_{N}$}
    \label{Alg:SJT}
\end{algorithm}


Initialize $\pi$ to the identity permutation, initialize $currentElement$ to $2$, initialize 
$N$ to $\pi_{max}$ and initialize all indecies of $direction$ to true. If $currentElement$ is greated than $N$, print $\pi$ 
and return. Otherwise, begin a for loop that runs $i=[1 \dots N-1]$ times. 
In the for loop, first make a recursive call with $currentElement$ increasing by one. 
If $direction[currentElement]$ is true, then swap $currentElement$ in $\pi$ with its left neighbor. Else 
$direction[currentElement]$ is false then swap the $currentElement$ in $\pi$ with its right neighbor.
Once the for loop has exited, make one more recursive call with $currentElement+1$ outside the for loop; 
this avoids an extra swap operation from occurring while still maintaining the correct number of recursive 
calls. Lastly, negate $direction[currentElement]$, which effectively changes the direction of the $currentElement$ 
for the next time it is to be swapped.

\begin{lemma}
    The time complexity of $SJT$ is $O(N!)$.
\end{lemma}
\begin{proof}
    The algorithm lists all $N!$ permutations. Given a permutation of order $N-1$,
    $N$ permutations of order $N$ are derived from this given permutation. Thus, the recurrence relation for $SJT(N)=N(SJT(N-1))$ with $SJT(1)=1$. 
    If $currentElement = N$ then each recursive call outputs a new permutation. If $currentElement < N$ then the number of 
    recusive calls that need to be made before a permutation is ouput is $N-currentElement+1$.
\end{proof}



%%Paragraph 
\subsection{Effler Rusky Algorithm}

In their paper, \textbf{A CAT Algorithm for Generating Permutations with a Fixed Number of Inversions}, the authors provide a 
\emph{constant amortized time} algorithm for generating all permutations of order $N$ witk $k$ inversions \cite{A26}. \emph{Constant amortized time}
refers to the average time taken per operation over many operations, given that worst case scenarios are rare. 
The algorithm is also a \emph{BEST} algorithm (backtracking esnures success at terminals), meaning that when the algorithm backtracks, 
the back-tracking leads to a successful result \cite{A26}. The algorithm moves from right to left and 
is based off of placing  an element $x \in [1 \dots N]$ at the next index in $\pi$ from right to left. The placement of $x$ in $\pi$ 
reduces $k$. Let $Pos(x)$ be the position of element $x$ in an ordered list of remaining elements of $[N]$. The remaining elements 
are defined as the elements of $[N]$ that have yet to be added to $\pi$.
For example,  let $N=4$. Let the current state of $\pi=(\underline{},\underline{},\underline{},2)$. Let the remaining elements be ascending in $L=[1 \dots 4] - \{2\}$. 
Assume element $x=3$ is to be placed at position $N'=3$ in $\pi$. Assume the current value of $K=2$. 
$Pos(3)$ in $L$ is two. Thus, $k$ is reduced by $N'-Pos(x)=3-2=1$; $K = K -1=1$. 
Meaning that by placing element $3$ in position three in $\pi$, the remaining number of inversions to be created in $\pi$ is $K=1$.
To see the algorithm please refer to Alg.\ref{Alg:KInv}. 
\begin{algorithm}
    \begin{algorithmic}[1]
        \Function{KInversions}{$\pi$, $N$, $k$, $list$}
            \If{$N = 0$ and $k = 0$}
                \State $Print(\pi)$
            \Else
                \For{$x \in list$}
                    \If{$N - pos(x) \leq k \leq {N-1\choose k} + n - pos(x)$}
                        $\pi_{N} \gets x$
                        \State remove $x$ from $list$ 
                        \State $KInversions(\pi, N-1, k \gets k - (N - Pos(x)), list)$
                        \State insert $x$ in $list$ at correct position.
                    \EndIf
                \EndFor
            \EndIf
        \EndFunction
        
    \end{algorithmic}
    \caption{Generate all permutations with $k$ inversions}
    \label{Alg:KInv}
\end{algorithm}

