\subsection{Ladders and Adjacent Transpositions}
A ladder lottery is a way of sorting a permutation, yet it can also be thought of as 
a decomposition of a permutation into \emph{adjacent transpositions}. \cite{A1} 
An \emph{adjacent transposition} is simply a swap of two adjacent elements in a 
permutation. For example, given the permutation (1, 3, 4, 2), an adjacent 
transposition could be done on the following pairs of elements: 
(1, 3), (3, 4) or (4, 2). Each would result in a unique permutation. 
Simply put, given any arbitrary starting permutation, $\pi$, keep swapping 
adjacent inversions until the identity permutation is reached.  An optimal 
ladder lottery from $\pi's$ optimal ladder set is a minimal sequence of 
adjacent transpositions such that $\pi$ is sorted into the identity permutation; 
each ladder in the set represents a sequence of adjacent transpositions for 
sorting $\pi$ into the identity permutation. For example, given the permutation 
(4, 3, 2, 1) there exists eight ladders in this permutation's optimal ladder set. 
Two of these ladders are found in \ref{fig:ac}:

\begin{figure}[!htp]
    \label{fig:ac}
	\begin{minipage}{0.4\textwidth}
		\centering
	
		\begin{tikzpicture}
			\draw(0, 0) to (0, 4) node[above]{4};
			\draw(2, 0) to (2, 4) node[above]{3};
			\draw(4, 0) to (4, 4) node[above]{2};
			\draw(6, 0) to (6, 4) node[above]{1};
			
			\draw(0, 3.7) to (2, 3.7);
				\draw node at (1, 3.9) {(4, 3)};
			\draw(2, 3.25) to (4, 3.25);
				\draw node at (3, 3.45){(4, 2)};
			\draw(4, 2.75) to (6, 2.75);
				\draw node at (5, 3.0){(4, 1)};
			
			\draw(0, 2.75) to (2, 2.75);
				\draw node at (1, 3.0){(3, 2)};
			\draw(2, 2.25) to (4, 2.25);
				\draw node at (3, 2.5){(3, 1)};
			
			
			\draw(0, 1.75) to (2, 1.75);
				\draw node at (1, 1.95){(2, 1)};
			
			\draw node at (0, -0.5){1};
			\draw node at (2, -0.5){2};
			\draw node at (4, -0.5){3};
			\draw node at (6, -0.5){4};
			
			%%second ladder%%
			\draw(9, 0) to (9, 4) node[above]{4};
			\draw(11, 0) to (11, 4)node[above]{3};
			\draw(13, 0) to (13, 4)node[above]{2};
			\draw(15, 0) to (15, 4)node[above]{1};
			
			\draw(9, 3.7) to (11, 3.7);
				\draw node at (10, 3.9) {(4, 3)};
			\draw(11, 3.25) to (13, 3.25);
				\draw node at (12, 3.45){(4, 2)};
			\draw(13, 2.75) to (15, 2.75);
				\draw node at (14, 3.0){(4, 1)};
			
			\draw(9, 1.25) to (11, 1.25);
				\draw node at (10, 1.5){(3, 1)};
			\draw(11, 2) to (13, 2);
				\draw node at (12, 2.25){(2, 1)};
			
			\draw(11, 0.65) to (13, 0.65);
				\draw node at (12, 0.85){(3, 2)};
			 
			
			\draw node at (9, -0.5){1};
			\draw node at (11, -0.5){2};
			\draw node at (13, -0.5){3};
			\draw node at (15, -0.5){4};	
		\end{tikzpicture}
	
	\end{minipage}
	

	
		
	\caption{The left ladder is one of eight unique ladders from (4,3,2,1)'s optimal ladder set. The right ladder is another one of eight unique ladders form (4,3,2,1)'s optimal ladder set}
		
\end{figure}

From looking at the above ladders, going from top left to bottom right, the left ladder represents the sequence of adjacent transpositions (4,3), (4,2), (4,1),(3,2),(3,1),(2,1) 
whereas the right ladder represent the sequence of adjacent transpositions 
(4, 3),(4, 2),(4, 1),(2, 1),(3, 1),(3, 2). 
Notice how the length of the sequences are the same,because both lengths are equal 
to the minimal number of swaps to sort (4, 3, 2, 1) 
it is simply the order in which the adjacent transpositions occur in the sequence 
that makes the sequences different from each other. 