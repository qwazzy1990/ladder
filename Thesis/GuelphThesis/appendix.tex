% This is an appendix, it behaves like an ordinary chapter
% appendices are nice places to put things like proofs that are
% too ugly to keep in the body of the thesis and things like
% code or data that no one really wants to look at


%-----------------------------------------------------
\chapter{Appendix}
%%\VerbatimInput{TestFile.txt}
%-----------------------------------------------------
% 4     3     2     1\newline     
% $\vert$-----$\vert$     |-----|\newline

%%\dividepage
%%\chapter{Extra Material}
%%\VerbatimInput{MinLaddersPrint.txt}
%%\input{MinLaddersPrint.txt}
\section{Table of $OptL\{\pi\}$ for all $\pi$ of order $5$}
\input{textTable.txt}
\pagebreak
\section{Code for {\sc CreateCanonical}}
\begin{singlespace}
\begin{footnotesize}
\begin{code}

#include <stdio.h>
#include <stdlib.h>
#include <string.h>


//ladder data structure
typedef struct ladder
{
    int** ladder;
    int numRows;
    int numCols;
    int numBars;
} Ladder;

//initialize the ladder 
Ladder* initLadder(int n);

//returns the coordinates of the bar or absence of a bar formed between x and y
int* GetCoordinates(int n, int x, int posX, int posY);
//creates a copy of a permutation
int* copyPerm(int* og, int start, int end);
//creates a copy of og permutation minus the element at index
int* removeElement(int* og, int index, int n);
//creates the canonical ladder
void CreateCanonical(Ladder* l, int* pi, int n);
//prints the ladder
void printLadder(Ladder *l);


//initializes the ladder data structure
//n is the length of a permutation
Ladder* initLadder(int n)
{
	Ladder* l = calloc(1, sizeof(Ladder));
	int nCols = n-1;
	int nRows = 2*(n-1) - 1;
	int nBars = 0;
	l->ladder = calloc(nRows, sizeof(int*));
	for(int i = 0; i < nRows; i++)l->ladder[i] = calloc(nCols, sizeof(int));
	l->numRows = nRows;
	l->numCols = nCols;
	l->numBars = nBars;
	return l;
}//end function

/**
	Returns the coordinates corresponding to a bar or absence of a bar 
	formed between elements x and y which lies on the associated diagonal of 'x'. 
	Parameter conditions: n is the size of a permutation, x is the element whose diagonal we are populating, posX is the position of x in pi, 
							posY is the position of 'y' in pi. posX != posY and x > y
    Post conditions: The (row, column) coordinates of the bar or absence of a bar corresponding to the two elements x and y
**/ 

int* GetCoordinates(int n, int x, int posX, int posY){
	
	int* coordinates = calloc(2, sizeof(int));
	//if x is to the right of y, then they form the absence of an inversion. Use 
	//equation 3.1.1 from Equation 3.1
	if(posX > posY){
		int row = (2*n - x - 1) - (x - posY) + 1;
		int col = (x - 1) - (x - posY) + 1;
		coordinates[0] = row ;
		coordinates[1] = col;
	}//end If
	//else if posX is < posY, then they form an inversion. Use equation 3.1.2 from Equation 3.1
	else{
		int row = (2*n - x - 1) - (x - posY);
		int col = (x - 1) - (x - posY);
		coordinates[0] = row;
		coordinates[1] = col;
	}//end else 

	return coordinates;

}//end function

//creates a copy of a permutation from a start index to an end index
int* copyPerm(int* og, int start, int end){
	
	int size = (end - start);
	int* copy = calloc(size, sizeof(int));
	for(int i = start; i < end; i++){
		copy[i - start] = og[i];
	}
	return copy;
}

int* removeElement(int* og, int index, int n){

	int* leftPart = copyPerm(og, 0, index);
	int* rightPart = copyPerm(og, index+1, n);
	int size = n-1;
	int* newPerm = calloc(n-1, sizeof(int));
	for(int i = 0; i < index; i++)newPerm[i] = leftPart[i];
	for(int i = 0; i < n-index-1; i++)newPerm[index+i] = rightPart[i];
	free(leftPart);
	free(rightPart);
	return newPerm;
}

/**
Sets ladder l to the Canical ladder. 
Parameter conditions: l is created using initLadder, pi is a permutation of order n
Post conditions: l is set to the canonical ladder corresponding to pi
**/
void CreateCanonical(Ladder* l, int* pi, int n)
{
	int x = n;
	int* pi2 = copyPerm(pi, 0, n);
	//from element x=n down to element x=2
	while(x > 1)
	{
		int index = -1;
		//get the index of element 'x' in pi
		for(int i = 0; i < x; i++){
			if(pi2[i] == x){
				index = i;
				break;
			}//end if
		}//end for
		//populate the associated diagonal of x with 1s and 0s.
		for(int i = 0; i < x; i++){
			if(i != index){
				int* coordinates = GetCoordinates(n, x, index, i);
				int row = coordinates[0];
				int col = coordinates[1];
				if(i < index){
                    l->ladder[row][col] = 0;
				}else{
                    l->ladder[row][col] = 1;
					l->numBars++; 
				}
			}
		}//end for
		pi2 = removeElement(pi2, index, x);
        x--;
	}//end while
}

void printLadder(Ladder *l)
{
   
    int n = l->numCols + 1;
    int offset = 2*(n-1) - 1;
    printf("\n\n");
    for (int i = 0; i < offset; i++)
    {
        for (int j = 0; j < l->numCols; j++)
        {
            printf("|");
            if (l->ladder[i][j] == 1)
                printf("-----");
            else
                printf("     ");
        }
        printf("|");
        printf("\n");
    }
}


//main for testing
int main(int argc, char* argv[])
{
    //testing permutation
    int pi[] = {4,1,7,2,3,6,5};
    int n = 7;
    Ladder* l = initLadder(n);
    CreateCanonical(l, pi, n);
    printLadder(l);
}

\end{code}
\end{footnotesize}
\end{singlespace}

\section{Code for {\sc ModifiedSJT}}

\begin{singlespace}
\begin{footnotesize}
\begin{code}
//Gets the coordinates of the row and coulumn for ModifiedSJT
int* GetCoordinatesTwo(int* rX, int n, int x, int posX, int posY);


//modified SJT algorithm for listing Ln
void ModifiedSJT(Ladder* l, int x, int n, bool* direction, int* rX);

//used in ModifiedSJT to return the location of a 1 or 0 indicating the addition or removal of 
//a bar between two adjacent elements, x>y
int* GetCoordinatesTwo(int* rX, int n, int x, int posX, int posY)
{
	int* coordinates = calloc(2, sizeof(int));
	//if posX is to the right of posY then x and y are currently 
	//uninverted and will be inverted. Therefore, we add a bar along the associated diagonal of 'x'
	if(posX == posY + 1)
	{
		coordinates[0] = (2*n - x - 1) - (*rX)-1;
		coordinates[1] = (x-1) - (*rX)-1;
		(*rX)++;

	}
	//else if x is to the left of y by 1 position, then x and y are currently inverted and will become 
	//uninverted. Therefore, we remove a bar along the associated diagnoal of 'x'
	else if(posX == posY - 1)
	{

		coordinates[0] = (2*n - x) - (*rX)-1;
		coordinates[1] = x - (*rX)-1;
		(*rX)--;

	}
	//else error occured
	else 
	{
		printf("Error occurred\n");
		exit(0);
	}

	return coordinates;
}

/** 
Modify the SJT algorithm to list Ln by adding or removing a bar
**/
void ModifiedSJT(Ladder* l, int x, int n, bool* direction, int* rX)
{
	if(x > n)
	{
		if(l->numBars == 0)printLadder(l);
		return;
	}
	for(int i = 0; i < x; i++)
	{
		if(i == 0)ModifiedSJT(l, x+1, n, direction, rX);
		else 
		{
			//adding to the associated diagonal of x
			if(direction[x-1] == false)
			{
				int* coordinates = GetCoordinatesTwo(&(rX[x-1]), n, x, 1, 0);
				//subtract 1 for 0 index
				int row = coordinates[0];
				int column = coordinates[1];

				l->ladder[row][column] = 1;
				l->numBars++;

			}
			//removing a bar from the associated diagonal of x
			else{
				int* coordinates = GetCoordinatesTwo(&(rX[x-1]), n, x, 0, 1);
				int row = coordinates[0];
				int column = coordinates[1];


				l->ladder[row][column] = 0;
				l->numBars--;	
			}
			printLadder(l);
			ModifiedSJT(l, x+1, n, direction, rX);


		}
	}
	direction[x-1] = !(direction[x-1]);
}


//testing main
int main(int argc, char* argv[])
{
  
    int n = 5;
    Ladder* l = initLadder(n);
   

	int* rX = calloc(n, sizeof(int));
	bool* direction = calloc(n, sizeof(int));
	for(int i = 0; i < n; i++){
		direction[i] = false;
		rX[i] = 0;
	}
	ModifiedSJT(l, 2, n, direction, rX);
}




\end{code}
\end{footnotesize}
\end{singlespace}


% \section{{\sc FindAllChildren}}

% \begin{algorithm}[!htp]
% 	\begin{algorithmic}[1]
% 		\Function{FindAllChildren}{$ladder$, $cleanLevel$, $n$}
% 			\State $currentRoute \gets n$
% 			\While{$currentRoute \geq cleanLevel$}
% 				\State going top left to bottom right 
% 				\For{$bar \in currentRoute$}
% 					\State $row \gets$ row of $bar$ in $ladder$ 
% 					\State $col \gets$ col of $bar$ in $ladder$
% 					\State $lowerNeighbor \gets ladder[row-1][col]$
% 					\If{$lowerNeighbor$ is right swappable}
% 						\State {\sc RightSwap($ladder$, $bar$, $lowerNeighbor$)}
% 						\State {\sc {\sc FindAllChildren}($ladder$, $y+1$, $n$)}
% 						\State {\sc LeftSwap($ladder$, $bar$, $lowerNeighbor$)}
% 					\EndIf
% 				\EndFor
% 				\State $currentRoute \gets currentRoute-1$
% 			\EndWhile
% 			\State $currentRoute \gets cleanLevel-1$
% 			\For{$bar \in currentRoute$}
% 				\State $row \gets$ row of $bar$ in $ladder$ 
% 				\State $col \gets$ col of $bar$ in $ladder$
% 				\State $lowerNeighbor \gets ladder[row-1][col]$
% 				\If{$lowerNeighbor$ is right swappable \textbf{and} is the rightmost bar of $currentRoute-1$}
% 					\State {\sc RightSwap($ladder$, $bar$)}
% 					\State {\sc FindAllChildren($ladder$, $cleanLevel$, $n$)}
% 					\State {\sc LeftSwap($ladder$, $bar$)}
% 				\EndIf
% 			\EndFor
% 		\EndFunction
% 	\end{algorithmic}
% 	\caption{{\sc FindAllChildren} which lists $OptL\{\pi\}$.}
% 	\label{Alg:FindAllChildren}
% \end{algorithm}

% One will note that two helper functions, {\sc RightSwap} and {\sc LeftSwap}, are required to complete 
% {\sc FindAllChildren}. The details of these algorithms are not found 
% in the paper~\cite{A1}. Thus, these algorithms and their details can be found in the Appendix in Algorithm~\ref{Alg:RightSwap}
% and Algorithm~\ref{Alg:LeftSwap}. {\sc FindAllChildren} is the first algorithm for generating $OptL\{\pi\}$.
% {\sc FindAllChildren} enumerates $OptL\{\pi\}$ as a tree of ladders.

% \section{Subroutines}

% The parent-child relation in the tree structure for {\sc FindAllChildren} is based on a \emph{local swap operation} 
% which corresponds to a braid relation and is a local modification of $ladder$ as shown in Figure~\ref{fig:rightSwap}.
% To get from the parent to the child a right swap operation is performed. To get from the child to the parent 
% a left swap operation is performed. 
% Given an arbitrary bar, $x$, it can be right swapped if and only if there are two bars, $y,z$ where $y \neq z$ 
% such that all the following conditions are met~\cite{A1}.
% \begin{itemize}
% 	\item The left end point of $z$ is directly above the left end point of $x$.
% 	\item The left end point of $y$ is directly above the right end point if $x$.
% 	\item The right end point of $z$ is directly above the left end point of $y$.
% \end{itemize}

% Given an arbitrary bar, $x$, it can be left swapped if and only if there are two bars, $y,z$ where $y \neq z$ 
% such that the following conditions are met~\cite{A1}.
% \begin{itemize}
% 	\item The right end point of $z$ is directly below the right end point of $x$.
% 	\item The right end point of $y$ is directly below the left end point if $x$.
% 	\item The left end point of $z$ is directly below the right end point of $y$.
% \end{itemize}
% In the left ladder in Figure~\ref{fig:rightSwap} bar $x=(3,1)$, bar $y=(5,1)$ and bar $z=(5,3)$. Bar $x$ can be right swapped 
% seeing as the three conditions for performing a right swap operation are met.
% In the right ladder in Figure~\ref{fig:rightSwap} bar $x=(3,1)$, bar $y=(5,1)$ and bar $z=(5,3)$. Bar $x$ can be left swapped 
% seeing as the three conditions for performing a left swap operation are met.\par
% \begin{figure}[t]
% 	\begin{minipage}{0.4\textwidth}
% 		\begin{center}

% 			%%drawing the lines
% 			\begin{tikzpicture}
% 				\draw(0, 0) to (0, 4) node[above]{5};
% 				\node at (0, -0.5){1};

% 				\draw(1, 0) to (1, 4) node[above]{2};
% 				\node at (1, -0.5){2};


% 				\draw(2, 0) to (2, 4) node[above]{3};
% 				\node at (2, -0.5){3};

% 				\draw(3, 0) to (3, 4) node[above]{1};
% 				\node at (3, -0.5){4};


% 				\draw(4, 0) to (4, 4) node[above]{4};
% 				\node at (4, -0.5){5};

% 				%%drawing the bars

% 				%%5's route
% 				\draw(0, 3.5)to (1, 3.5);
% 					\draw node at (0.5, 3.8) {(5,2)};
% 				\draw[line width=1mm, red](1, 3) to (2, 3);
% 					\draw node at (1.5, 3.3) {(5,3)};
% 				\draw[line width=1mm, red](2, 2.5) to (3, 2.5);
% 					\draw node at (2.5, 2.8) {(5,1)};
% 				\draw(3, 2) to (4, 2);
% 					\draw node at (3.5, 2.3) {(5,4)};

% 				%%4's route, no bars

% 				%%3s route
% 				\draw[line width=1mm, red](1, 2) to (2, 2);
% 					\draw node at (1.5, 2.3) {(3,1)};
% 				%%2s route
% 				\draw(0, 1.5) to (1, 1.5);
% 					\draw node at (0.5, 1.8){(2,1)};

% 			\end{tikzpicture}
% 		\end{center}
% 	\end{minipage}
% 	\begin{minipage}{0.4\textwidth}
% 		\begin{flushright}

% 			%%drawing the lines
% 			\begin{tikzpicture}
% 				\draw(0, 0) to (0, 4) node[above]{5};
% 				\node at (0, -0.5){1};

% 				\draw(1, 0) to (1, 4) node[above]{2};
% 				\node at (1, -0.5){2};


% 				\draw(2, 0) to (2, 4) node[above]{3};
% 				\node at (2, -0.5){3};

% 				\draw(3, 0) to (3, 4) node[above]{1};
% 				\node at (3, -0.5){4};


% 				\draw(4, 0) to (4, 4) node[above]{4};
% 				\node at (4, -0.5){5};

% 				%%Drawing the bars
% 				\draw(0, 3.5)to (1, 3.5);
% 					\draw node at (0.5, 3.8){(5,2)};
% 				\draw[line width=1mm, red](2, 3.5) to (3, 3.5);
% 					\draw node at (2.5, 3.8) {(3,1)};
% 				\draw[line width=1mm, red](2, 2.5) to (3, 2.5);
% 					\draw node at (2.5, 2.8) {(5,3)};
% 				\draw(3, 2) to (4, 2);
% 					\draw node at (3.5, 2.3){(5,4)};
% 				%%4's route, no bars

% 				%%3s route
% 				\draw[line width=1mm, red](1, 3) to (2, 3);
% 					\draw node at (1.5, 3.3) {(5,1)};
% 				%%2s route
% 				\draw(0, 2) to (1, 2);
% 					\draw node at(0.5, 2.3){(2,1)};
% 			\end{tikzpicture}
% 		\end{flushright}
% 	\end{minipage}
% 	\caption{Example of a local swap operation. When a right swap operation is performed
% 	on the left ladder, the result is the right ladder. When a left swap operation is performed
% 	on the right ladder, the result is the left ladder.}
% 	\label{fig:rightSwap}
% \end{figure}
% The following algorithms are used to perform a local swap operation: 
% Algorithm {\sc RightSwap}, Algorithm {LeftSwap} and Algorithm {\sc ShiftSubLadder}. 
% Let $ladder$ be initialized to an empty ladder. Let $bar$ be the bar to be right or left swapped in the $ladder$.
% In Algorithm {\sc ShiftSubLadder}, \textit{offset} is initialized to $2$ when right swapping and initialized to 
% $-2$ when left swapping. In Algorithm {\sc ShiftSubLadder} \textit{index} is initialized to $1$ when 
% right swapping and $-1$ when left swapping.
% Let the \emph{right child} of some arbitrary bar $w$, $rc(w)$ for short, be the bar one row below, and one column to the 
% right of $w$. Let the \emph{left child} of some arbitrary bar $w$, $lc(w)$ for short, be the bar one 
% row below, and one column to the left of $w$. 
% Let the \emph{right sibling} of $w$, $rs(w)$ for short,  be defined as the bar on the same row as some arbitrary bar $w$ and
% two columns to the right of bar $w$. 
% Let the \emph{upper neighbor} of $w$, $un(w)$ for short, be the bar that is two rows above the row of $w$ and is 
% in the same column as $w$. Let the \emph{right neighbor} of $w$, $rn(w)$ for short, be the bar that is one row 
% above $w$ and one column to the right of $w$. Let the lower neighbor of $w$, $ln(w)$ for short be 
% the bar that is two rows below $w$ and in the same column as $w$. Let the left neighbor of 
% $w$, $lfn(n)$ for short, be the bar one row below and one column to the left of $w$.\pagebreak
% %%%%%%% ALGORITHM RIGHT SWAP %%%%%%%%%%%%%%%%%%%
% \begin{algorithm}[t]
% 	\begin{algorithmic}[1]
% 		\Function{RightSwap}{$ladder$, $bar$}
% 			\State $row \gets bar's$ row in $ladder$
% 			\State $col \gets bar's$ column in $ladder$
% 			\State $upperNeighbor \gets un(bar)$
% 			\State $rightNeighbor \gets rn(bar)$
% 			\If{$col<=n-3$}
% 				\State $subLadderRoot \gets$ $rs(bar)$
% 				\State {\sc ShiftSubLadder($ladder$, $subLadderRoot$, 2)}
% 			\EndIf
% 			\State {\sc Swap($upperNeighbor$, $ladder[row+1][col+1]$)}
% 			\State {\sc Swap($bar$, $rightNeighbor$)}
% 		\EndFunction
% 	\end{algorithmic}
% 	\caption{Perform a right swap operation on a bar}
% 	\label{Alg:RightSwap}
% \end{algorithm}

% %%%%%%%%%% ALGORITHM LEFT SWAP %%%%%%%%%%%%%%%%%%%%%%%%
% \begin{algorithm}[!htp]
% 	\begin{algorithmic}[1]
% 		\Function{LeftSwap}{$ladder$, $bar$}
% 			\State $row \gets bar's$ row in ladder 
% 			\State $col \gets bar's$ column in ladder
% 			\State $leftNeighbor \gets lfn(bar)$
% 			\State {\sc Swap($bar$, $leftNeighbor$)}
% 			\State $lowerNeighbor \gets ln(bar)$
% 			\If{$col<n-1$}
% 				\State $subLadderRoot \gets$ $rc(lowerNeighbor)$
% 				\State {\sc Swap($lowerNeighbor$, $ladder[row-1][col-1]$)}
% 				\State {\sc ShiftSubLadder($ladder$, $subLadderRoot$, $-2$)}
% 			\Else 
% 				\State {\sc Swap($lowerNeighbor$, $ladder[row-1][col-1]$)}
% 			\EndIf
% 		\EndFunction
% 	\end{algorithmic}
% 	\caption{Perform a left swap operation on a bar}
% 	\label{Alg:LeftSwap}
% \end{algorithm}\pagebreak

% %%%%%%%%%%%% ALGORITHM ShiftSubLadderDOWN %%%%%%%%%%%%%%%%%%%%%%
% \begin{algorithm}[t]
% 	\begin{algorithmic}[1]
% 		\Function{ShiftSubLadder}{$ladder$, $bar$, \textit{offset}}
% 			\State $row \gets bar's$ row in ladder
% 			\State $col \gets bar's$ column in ladder
% 			\State $rightChild \gets rc(bar)$
% 			\State $leftChild \gets lc(bar)$
% 			\If{$rightChild=0$ \textbf{and} $leftChild=0$}
% 				\State {\sc Swap($ladder[row+$\textit{offset}$][col]$, $ladder[row][col]$)}
% 				\State \textbf{return}
% 			\Else 
% 				\If{\textit{offset}$==-2$ indicating a left swap}
% 					\State {\sc Swap($ladder[row+$\textit{offset}$][col]$, $ladder[row][col]$)}
% 					\If{$rightChild \neq 0$}
% 						\State {\sc ShiftSubLadder($ladder$, $rightChild$, \textit{offset})}
% 					\EndIf 
% 					\If{$leftChild \neq 0$}
% 						\State {\sc ShiftSubLadder($ladder$, $leftChild$, \textit{offset})}
% 					\EndIf
% 				\EndIf
% 				\If{\textit{offset}$==2$ indicating a right swap}
% 					\If{$rightChild \neq 0$}
% 						\State {\sc ShiftSubLadder($ladder$, $rightChild$, \textit{offset})}
% 					\EndIf 
% 					\If{$leftChild \neq 0$}
% 						\State {\sc ShiftSubLadder($ladder$, $leftChild$, \textit{offset})}
% 					\EndIf 
% 					\State {\sc Swap($ladder[row+$\textit{offset}$][col]$, $ladder[row][col]$)}
% 				\EndIf
% 			\EndIf
% 		\EndFunction
% 	\end{algorithmic}
% 	\caption{Shifts the sub-ladder up or down the ladder data structure depending on if a right or left swap operation is being performed}
% 	\label{Alg:ShiftSubLadder}
% \end{algorithm}

% The way that the aforementioned three algorithms work in order 
% to complete a local swap operation is as follows. 
% When performing a right swap operation, {\sc RightSwap}
% takes the current bar, $x$, and gets its upper neighbor $z$ and its right neighbor $y$; $x$, $z$ and $y$ meet the 
% criteria for performing a right swap operation. Then, {\sc RightSwap} calls {\sc ShiftSubLadder}
% with the \emph{offset} value of $2$ and the \emph{index} value of one. 
% Algorithm {\sc ShiftSubLadder} ensures that the bottom right sub-ladder, beginning at the right sibling of $x$/$rs(x)$, 
% is shifted down the ladder such that when the right 
% swap operation is performed, the root of the sub-ladder is the right child of $z$. 
% When performing a right swap, {\sc ShiftSubLadder} moves each bar in the sub-ladder two rows 
% down the ladder. Since each bar in the sub-ladder is a left or right child of some other bar in the sub-ladder, 
% with the exception of the root ladder, the index is set to $1$ indicating the offset.
% When a right swap operation is about to occur, bar $z$ will be moved from its current row and column to its current row + $3$ 
% and its current column $+1$. Once the right swap operation is performed, 
% $rs(x)$ becomes $rc(z)$. $y$ and $x$ are swapped. Then the function is complete.\par 
% The left swap operation reverses the right swap operation; {\sc LeftSwap}
% is the inverse of {\sc RightSwap}. The \textit{offset} value for {\sc ShiftSubLadder} when 
% performing a left swap is set to $-2$ to indicate the bars need to be moved up the ladder. 
% When bars are moved in {\sc LeftSwap}, the parent bar is shifted upward before its children.
% This is unlike the {\sc RightSwap} in which the children bars are shifted downward 
% before their parent. This is done to ensure that for any two bars in the sub-ladder, 
% $x,y$, $x$ will not be swapped with $y$, thus putting $y$ in the wrong position. 
% This is why the $lowerNeighbor$ of $x$ in {\sc LeftSwap} is swapped before 
% {\sc ShiftSubLadder} is called. Whereas in {\sc RightSwap} the $upperNeighbor$ 
% of $x$ is swapped after {\sc ShiftSubLadder} is called.
% To see an example of all three algorithms performing a local swap operation please refer to Figure \ref{Fig:SwapAndShift}.\pagebreak
% \begin{figure}[ht]
% 	\begin{minipage}{.4\textwidth}
% 		\begin{tikzpicture}
% 			\node at(2, 3.5){\small{L1: Original Ladder}};
% 			\draw(0, 0) to (0, 3);
% 					\node at(.5, 2.9){\small{$z$}};
% 				\draw(0, 2.7) to (1,2.7);
% 					\node at(.5, 2.1){\small{$x$}};

% 				\draw(0,1.9) to (1,1.9);
% 				\draw(0,1.1) to (1,1.1);
% 				\draw(0,0.3) to (1,0.3);
% 			\draw(1, 0) to (1, 3);
% 				\node at(1.5,2.5){\small{$y$}};
% 				\draw(1,2.3) to (2,2.3);
% 				\draw(1,1.5) to (2,1.5);
% 				\draw(1,0.7) to (2,0.7);
% 			\draw(2, 0) to (2, 3);
% 				\node at(2.5, 2.1){\tiny{$rs(x)$}};
% 				\draw(2,1.9) to (3,1.9);
% 				\draw(2,1.1) to (3,1.1);
% 			\draw(3, 0) to (3, 3);
% 				\draw(3,1.5) to (4,1.5);
% 			\draw(4, 0) to (4, 3);

% 			\node at(-1, 2.7){\small{$R1$}};
% 			\node at(-1, 2.3){\small{$R2$}};
% 			\node at(-1, 1.9){\small{$R3$}};
% 			\node at(-1, 1.5){\small{$R4$}};
% 			\node at(-1, 1.1){\small{$R5$}};
% 			\node at(-1, .7){\small{$R6$}};
% 			\node at(-1, .3){\small{$R7$}};

			
% 		\end{tikzpicture}
% 	\end{minipage}
% 	\begin{minipage}{.4\textwidth}
% 		\begin{tikzpicture}
% 			\node at(2, 3.5){\small{L2: Shift Sub-Ladder Down}};

% 			\draw(0, -1) to (0, 3);
% 					\node at(.5, 2.9){\small{$z$}};
% 				\draw(0, 2.7) to (1,2.7);
% 					\node at(.5, 2.1){\small{$x$}};

% 				\draw(0,1.9) to (1,1.9);
% 				\draw(0,.3) to (1,.3);
% 				\draw(0,-0.5) to (1,-0.5);
% 			\draw(1, -1) to (1, 3);
% 				\node at(1.5,2.5){\small{$y$}};
% 				\draw(1,2.3) to (2,2.3);
% 				\draw(1,0.7) to (2,0.7);
% 				\draw(1,-0.1) to (2,-0.1);
% 			\draw(2, -1) to (2, 3);
% 				\node at(2.5, 1.3){\tiny{$rs(x)$}};
% 				\draw(2,1.1) to (3,1.1);
% 				\draw(2,.3) to (3,.3);
% 			\draw(3, -1) to (3, 3);
% 				\draw(3,.7) to (4,.7);
% 			\draw(4, -1) to (4, 3);

% 			\node at(-1, 2.7){\small{$R1$}};
% 			\node at(-1, 2.3){\small{$R2$}};
% 			\node at(-1, 1.9){\small{$R3$}};
% 			\node at(-1, 1.5){\small{$R4$}};
% 			\node at(-1, 1.1){\small{$R5$}};
% 			\node at(-1, .7){\small{$R6$}};
% 			\node at(-1, .3){\small{$R7$}};
% 			\node at(-1, -.1){\small{$R8$}};
% 			\node at(-1, -.5){\small{$R9$}};


			
% 		\end{tikzpicture}
% 	\end{minipage}
% 	\begin{minipage}{.4\textwidth}
% 		\begin{tikzpicture}
% 			\node at(2, 3.5){\small{L3: Right Swap}};

% 			\draw(0, -1) to (0, 3);
% 					\node at(1.5, 1.7){\small{$z$}};
				
% 					\node at(.5, 2.1){\small{$y$}};

% 				\draw(0,1.9) to (1,1.9);
% 				\draw(0,.3) to (1,.3);
% 				\draw(0,-0.5) to (1,-0.5);
% 			\draw(1, -1) to (1, 3);
% 				\node at(1.5,2.5){\small{$x$}};
% 				\draw(1,2.3) to (2,2.3);
% 				\draw(1, 1.5) to (2,1.5);
% 				\draw(1,0.7) to (2,0.7);
% 				\draw(1,-0.1) to (2,-0.1);
% 			\draw(2, -1) to (2, 3);
% 				\node at(2.5, 1.3){\tiny{$rs(x)$}};
% 				\draw(2,1.1) to (3,1.1);
% 				\draw(2,.3) to (3,.3);
% 			\draw(3, -1) to (3, 3);
% 				\draw(3,.7) to (4,.7);
% 			\draw(4, -1) to (4, 3);

% 			\node at(-1, 2.7){\small{$R1$}};
% 			\node at(-1, 2.3){\small{$R2$}};
% 			\node at(-1, 1.9){\small{$R3$}};
% 			\node at(-1, 1.5){\small{$R4$}};
% 			\node at(-1, 1.1){\small{$R5$}};
% 			\node at(-1, .7){\small{$R6$}};
% 			\node at(-1, .3){\small{$R7$}};
% 			\node at(-1, -.1){\small{$R8$}};
% 			\node at(-1, -.5){\small{$R9$}};


			
% 		\end{tikzpicture}
% 	\end{minipage}
% 	\hfill
% 	\begin{minipage}{.4\textwidth}
% 		\begin{tikzpicture}
% 			\node at(2, 3.5){\small{L4: Left Swap}};

% 			\draw(0, -1) to (0, 3);
% 					\node at(0.5, 2.9){\small{$z$}};
				
% 					\node at(.5, 2.1){\small{$x$}};

% 				\draw(0,1.9) to (1,1.9);
% 				\draw(0,.3) to (1,.3);
% 				\draw(0,-0.5) to (1,-0.5);
% 			\draw(1, -1) to (1, 3);
% 				\node at(1.5,2.5){\small{$y$}};
% 				\draw(1,2.3) to (2,2.3);
% 				\draw(0, 2.7) to (1,2.7);
% 				\draw(1,0.7) to (2,0.7);
% 				\draw(1,-0.1) to (2,-0.1);
% 			\draw(2, -1) to (2, 3);
% 				\node at(2.5, 1.3){\tiny{$rs(x)$}};
% 				\draw(2,1.1) to (3,1.1);
% 				\draw(2,.3) to (3,.3);
% 			\draw(3, -1) to (3, 3);
% 				\draw(3,.7) to (4,.7);
% 			\draw(4, -1) to (4, 3);

% 			\node at(-1, 2.7){\small{$R1$}};
% 			\node at(-1, 2.3){\small{$R2$}};
% 			\node at(-1, 1.9){\small{$R3$}};
% 			\node at(-1, 1.5){\small{$R4$}};
% 			\node at(-1, 1.1){\small{$R5$}};
% 			\node at(-1, .7){\small{$R6$}};
% 			\node at(-1, .3){\small{$R7$}};
% 			\node at(-1, -.1){\small{$R8$}};
% 			\node at(-1, -.5){\small{$R9$}};
% 		\end{tikzpicture}
% 	\end{minipage}
% 	\begin{center}
% 	 \begin{minipage}{.5\textwidth}
% 	 	\begin{tikzpicture}
% 	 		\node at(2, 3.5){\small{L5=L1: Shift Sub-ladder Up}};
% 	 		\draw(0, 0) to (0, 3);
% 	 				\node at(.5, 2.9){\small{$z$}};
% 	 			\draw(0, 2.7) to (1,2.7);
% 	 				\node at(.5, 2.1){\small{$x$}};

% 	 			\draw(0,1.9) to (1,1.9);
% 	 			\draw(0,1.1) to (1,1.1);
% 	 			\draw(0,0.3) to (1,0.3);
% 	 		\draw(1, 0) to (1, 3);
% 	 			\node at(1.5,2.5){\small{$y$}};
% 	 			\draw(1,2.3) to (2,2.3);
% 	 			\draw(1,1.5) to (2,1.5);
% 	 			\draw(1,0.7) to (2,0.7);
% 	 		\draw(2, 0) to (2, 3);
% 	 			\node at(2.5, 2.1){\tiny{$rs(x)$}};
% 	 			\draw(2,1.9) to (3,1.9);
% 	 			\draw(2,1.1) to (3,1.1);
% 	 		\draw(3, 0) to (3, 3);
% 	 			\draw(3,1.5) to (4,1.5);
% 	 		\draw(4, 0) to (4, 3);

% 	 		\node at(-1, 2.7){\small{$R1$}};
% 	 		\node at(-1, 2.3){\small{$R2$}};
% 	 		\node at(-1, 1.9){\small{$R3$}};
% 	 		\node at(-1, 1.5){\small{$R4$}};
% 	 		\node at(-1, 1.1){\small{$R5$}};
% 	 		\node at(-1, .7){\small{$R6$}};
% 	 		\node at(-1, .3){\small{$R7$}};

		
% 	 	\end{tikzpicture}
% 	 \end{minipage}
% 	 \end{center}
% 	\caption{$x,y,z$ to be locally swapped. $rs(x)$ is the root of the sub-ladder.}
% 	\label{Fig:SwapAndShift}
% \end{figure}
% In concluding the section on {\sc FindAllChildren}, we have analyzed the original paper along with providing three subroutines 
% for the completion of {\sc FindAllChildren}.




